\documentclass{beamer}
%
% Choose how your presentation looks.
%
% For more themes, color themes and font themes, see:
% http://deic.uab.es/~iblanes/beamer_gallery/index_by_theme.html
%
\mode<presentation>
{
  \usetheme{default}      % or try Darmstadt, Madrid, Warsaw, ...
  \usecolortheme{default} % or try albatross, beaver, crane, ...
  \usefonttheme{default}  % or try serif, structurebold, ...
  \setbeamertemplate{navigation symbols}{}
  \setbeamertemplate{caption}[numbered]
} 

\usepackage[brazil]{babel}
\usepackage[utf8]{inputenc}
\usepackage[T1]{fontenc}

\usepackage{tikz}

\title[Determinantes]{Determinantes de Matrizes}
\author{MAP 2110 - Diurno}
\institute{IME USP}
\date{26 de maio}

\begin{document}

\begin{frame}
  \titlepage
\end{frame}

\begin{frame}{A importância de uma matriz invertível}
  $$ A\mathbf{x} = \mathbf{b} \implies x= A^{-1}\mathbf{b} $$

  \begin{center}
    \begin{tikzpicture}
      \draw[gray!50] (-0.5, -0.5) grid (3.5,3.5);
      \draw[red!50] (4.5,-0.5) grid (8.5,3.5);
      \draw[-latex, blue] (2,2.0) .. controls (3,3) and (5,3) .. node[anchor=south] {$A$} (6,2.0);
      \draw[-latex, red] (6,1.0) .. controls (5,0) and (3,0) .. node[anchor=north] {$A^{-1}$} (2,1);
      \coordinate (A) at (1.5,1.5);
      \draw (A) node {$\mathbb{R}^n$}; 
      \coordinate (B) at (6.5,1.5);
      \draw (B) node {$\mathbb{R}^n$};
      \end{tikzpicture}
  \end{center}

\end{frame}

\begin{frame}{Caso $2\times 2$}
\begin{gather*}
  A = \begin{pmatrix}
    a_{11} & a_{12} \\ 
    a_{21} & a_{22}
  \end{pmatrix} \text{ invertível } \iff \det{A} \neq 0 \\
  \det(A) = a_{11}a_{22} - a_{21}a_{12} \\
  \text{ de fato, se } \det(A)\neq 0 \implies A^{-1} = \frac{1}{\det(A)}\begin{pmatrix}
    a_{22} & -a_{12} \\ -a_{21} & a_{11}
  \end{pmatrix} 
\end{gather*}
\end{frame}

\begin{frame}{}
  Pra mostrar que se $A$ é invertível usamos a eliminação de Gauss, supondo $a_{11}\neq 0$:
  $$
  \begin{array}{|cc|c|}\hline
    a_{11} & a_{12} & \\ 
    a_{21} & a_{22} & \textcolor{red}{a_{11}L_2 - a_{21}L_1} \\ \hline  
  \end{array}\rightarrow
  \begin{array}{|cc|c|}\hline
    a_{11} & a_{12} & \\ 
    0 & \det(A) & \textcolor{red}{\leftarrow} \\ \hline  
  \end{array}
  $$
 
\end{frame}

\begin{frame}{Algumas propriedades do determinante (caso $2\times 2$)}
  Estas propriedades queremos estender para a os determinantes de dimensões maiores:
  \begin{enumerate}
    \item $\det(I)=1$ 
    \item Se uma coluna ou uma linha for zero então $det(A)=0$
    \item Se $A$ e $B$ diferem só por troca de linhas ou de colunas, então $\det(A) = -\det(B)$
    \item Se a primeira linha de $B$ é um $a$ vezes a primeira linha de $A$ e as segundas linhas são as mesmas, então 
    $\det(B) = a\det(A)$
    \item Se a diferença entre $A$ e $B$ é que a primeira linha de $B$ é a primeira linha de $A$ mais um multiplo da segunda linha de $A$ então $\det(A)=\det(B)$
  \end{enumerate}
\end{frame}
\begin{frame}{Comentários}
  
\end{frame}

\begin{frame}{Determinantes de matrizes $3\times 3$}
  Antes de dar uma definição do caso geral, trataremos o caso 
  $3\times 3$ para nos convencermos de que estamos no caminho certo.
  Queremos definir um número para a matriz, que indidique que ela é invertível.
  \begin{gather*}
    \begin{array}{|ccc|l|}\hline
      a_{11} & a_{12} & a_{13} & \\
      a_{21} & a_{22} & a_{23} & \textcolor{red}{a_{11}L_2 - a_{21}L1}\\
      a_{31} & a_{32} & a_{33} & \textcolor{red}{a_{11}L_3 - a_{31}L1}\\ \hline
      \end{array} \rightarrow \\
    \begin{array}{|ccc|l|}\hline
        a_{11} & a_{12} & a_{13} & \\
        0 & \textcolor{blue}{a_{11}a_{22}-a_{21}a_{12}} & \textcolor{blue}{a_{11}a_{23}-a_{21}a_{13}} & \\
        0 & \textcolor{blue}{a_{11}a_{32} -a_{31}a{12}} & \textcolor{blue}{a_{11}a_{33}-a_{31}a_{13}} & \\ \hline
    \end{array}
  \end{gather*}
  
\end{frame}

\begin{frame}
  \begin{align*}
    &\textcolor{blue}{(a_{11}a_{22}-a_{21}a_{12})}\textcolor{blue}{(a_{11}a_{33}-a_{31}a_{13})} \\
  - &\textcolor{red}{(a_{11}a_{23}-a_{21}a_{13})}\textcolor{red}{(a_{11}a_{32} -a_{31}a_{12})}
  \end{align*}
  Distribuindo a multiplicação:
  \begin{align*}
    &\textcolor{blue}{a_{11}a_{22}a_{11}a_{33}- a_{11}a_{22}a_{31}a_{13} -a_{21}a_{12}a_{11}a_{33}+ \fbox{$a_{21}a_{12}a_{31}a_{13}$}} \\
  - &(\textcolor{red}{a_{11}a_{23}a_{11}a_{32} -a_{21}a_{13} a_{11}a_{32} -a_{11}a_{23}a_{31}a_{12} +\fbox{$a_{21}a_{13}a_{31}a_{12}$}}) \\
  = & a_{11}(\textcolor{blue}{ a_{22}a_{11}a_{33} - 
  a_{22}a_{31}a_{13} -a_{21}a_{12}a_{33}}\textcolor{red}{-
  a_{23}a_{11}a_{32}+a_{21}a_{13}a_{32}+ \cdots }\\
  & \textcolor{red}{+ a_{23}a_{31}a_{12}}) \\
  = &
  a_{11}\left( a_{11}
   \begin{vmatrix}
    a_{22} & a_{23} \\
    a_{32} & a_{33}
  \end{vmatrix} -
   a_{21}
   \begin{vmatrix}
    a_{12} & a_{13} \\
    a_{32} & a_{33}
  \end{vmatrix} + a_{31}
  \begin{vmatrix}
    a_{12} & a_{13}\\
    a_{22} & a_{23}
  \end{vmatrix} \right) 
  \end{align*}
\end{frame}

\begin{frame}
  Para matrizes de dimensão $3\times 3$ podemos definir então:

  $$ \det(A) = \left( a_{11}
  \begin{vmatrix}
   a_{22} & a_{23} \\
   a_{32} & a_{33}
 \end{vmatrix} -
  a_{21}
  \begin{vmatrix}
   a_{12} & a_{13} \\
   a_{32} & a_{33}
 \end{vmatrix} + a_{31}
 \begin{vmatrix}
   a_{12} & a_{13}\\
   a_{22} & a_{23}
 \end{vmatrix} \right)$$
 Se $A =\begin{pmatrix}
   1 & 0 & 2 \\ 2 & 1 & -1 \\ 0 & -2 & 4 
 \end{pmatrix}$ então
 $$ \det(A)=\left( 1
   \begin{vmatrix}
    1 & -1 \\
    -2 & 4
  \end{vmatrix} -
   2
   \begin{vmatrix}
    0 & 2 \\
    -2 & 4
  \end{vmatrix} + 0
  \begin{vmatrix}
    0 & 2\\
    1 & -1
  \end{vmatrix} \right)=-6$$

\end{frame}

\begin{frame}{exemplos}
  Se $A =\begin{pmatrix}
    1 & 1 & 2 \\ 2 & 1 & -1 \\ 1& 1 & 2
  \end{pmatrix}$ 
  $$ \det(A)= \left( 1
  \begin{vmatrix}
   1 & -1 \\
   1 &  2
 \end{vmatrix} -
  2
  \begin{vmatrix}
   1 & 2 \\
   1 & 2
 \end{vmatrix} + 1
 \begin{vmatrix}
   1 & 2\\
   1 & -1
 \end{vmatrix} \right)=0$$

 Se $A =\begin{pmatrix}
0 & -2 & 4 \\ 2 & 1 & -1  \\1 & 0 & 2
\end{pmatrix}$ então
$$ \det(A)=\left( 0
  \begin{vmatrix}
   1 & -1 \\
   0& 2
 \end{vmatrix} -
  2
  \begin{vmatrix}
   -2 & 4 \\ 0 & 2 
 \end{vmatrix} + 1
 \begin{vmatrix}
   -2 & 4\\
   1 & -1
 \end{vmatrix} \right)=6$$
\end{frame}

\begin{frame}{A definição geral}
  Se $A$ é uma matriz quadrada de dimensão $n$, então, por 
  definição, $A_{ij}$ é a matriz quadrada de dimensão $n-1$
  obtida de $A$ eliminando-se a $i$-ésima linha e a
  $j$-ésima coluna.

  Se tivermos definido o determinante de matrizes 
  de dimensão $n-1$ vamos definir o cofator da posição
  $ij$ da matriz $A$ como o número
  $$ c_{ij}(A)=(-1)^{i+j}\det(A_{ij}) $$
  Iremos definir
  $$ \det{A} = \sum_{i=1}^n a_{i1}c_{i1}(A)$$
  
\end{frame}

\begin{frame}{Exemplos}
  Vamos ver a matriz $ A =\begin{pmatrix} 0 & -2 & 4 \\ 2 & 1 & -1  \\1 & 0 & 2 \end{pmatrix} $ .

\begin{gather*}
A_{11} = \begin{bmatrix}
  1 & -1 \\ 0 & 2
\end{bmatrix}\text{ }A_{12}=\begin{bmatrix}
  2 & -1 \\ 1 & 2
\end{bmatrix}\text{ }A_{13}=\begin{bmatrix}
  2 & 1 \\ 1 & 0
\end{bmatrix} \\
A_{21} = \begin{bmatrix}
  -2 & 4 \\ 0 & 2
\end{bmatrix}\text{ }A_{22}=\begin{bmatrix}
  0 & 4 \\ 1 & 2
\end{bmatrix}\text{ }A_{23}=\begin{bmatrix}
  0 & -2 \\ 1 & 0
\end{bmatrix} \\
A_{31} = \begin{bmatrix}
  -2 & 4 \\ 1 & -1
\end{bmatrix}\text{ }A_{32}=\begin{bmatrix}
  0 & 4 \\ 2 & {1}
\end{bmatrix}\text{ }A_{33}=\begin{bmatrix}
  0 & -2 \\ 2 & 1
\end{bmatrix}
\end{gather*}
\end{frame}

\begin{frame}{Exemplos}
Vamos ver a matriz $ A =\begin{pmatrix} 0 & -2 & 4 \\ 2 & 1 & -1  \\1 & 0 & 2 \end{pmatrix} $ .

\begin{gather*}
A_{11} = \begin{bmatrix}
1 & -1 \\ 0 & 2
\end{bmatrix}\text{ }A_{12}=\begin{bmatrix}
2 & -1 \\ 1 & 2
\end{bmatrix}\text{ }A_{13}=\begin{bmatrix}
2 & 1 \\ 1 & 0
\end{bmatrix} \\
A_{21} = \begin{bmatrix}
-2 & 4 \\ 0 & 2
\end{bmatrix}\text{ }A_{22}=\begin{bmatrix}
0 & 4 \\ 1 & 2
\end{bmatrix}\text{ }A_{23}=\begin{bmatrix}
0 & -2 \\ 1 & 0
\end{bmatrix} \\
A_{31} = \begin{bmatrix}
-2 & 4 \\ 1 & -1
\end{bmatrix}\text{ }A_{32}=\begin{bmatrix}
0 & 4 \\ 2 & \textcolor{blue}{-1}
\end{bmatrix}\text{ }A_{33}=\begin{bmatrix}
0 & -2 \\ 2 & 1
\end{bmatrix}
\end{gather*}
\end{frame}

\begin{frame}{Matriz de cofatores}
\begin{gather*}
  C=
\begin{bmatrix}
  \textcolor{blue}{ +
\begin{vmatrix}
1 & -1 \\ 0 & 2
\end{vmatrix}}&
\textcolor{red}{-\begin{vmatrix}
2 & -1 \\ 1 & 2
\end{vmatrix}}&
\textcolor{blue}{+\begin{vmatrix}
2 & 1 \\ 1 & 0
\end{vmatrix}} \\
\textcolor{red}{-\begin{vmatrix}
-2 & 4 \\ 0 & 2
\end{vmatrix}} & 
\textcolor{blue}{+\begin{vmatrix}
0 & 4 \\ 1 & 2
\end{vmatrix}} &
\textcolor{red}{-\begin{vmatrix}
0 & -2 \\ 1 & 0
\end{vmatrix}} \\
\textcolor{blue}{+\begin{vmatrix}
-2 & 4 \\ 1 & -1
\end{vmatrix}}&
\textcolor{red}{-\begin{vmatrix}
0 & 4 \\ 2 & {-1}
\end{vmatrix}}&
\textcolor{blue}{+\begin{vmatrix}
0 & -2 \\ 2 & 1
\end{vmatrix}}
\end{bmatrix}= \begin{bmatrix}
  2 & -5 & -1 \\ 4 & -4 & -2 \\ -2 & 8 & -4
\end{bmatrix}
\end{gather*}
\end{frame}

\begin{frame}
  $$
 \begin{bmatrix}
   0 & 2 & 1 \\ -2 & 1 & 0 \\ 4 & -1 & 2
 \end{bmatrix} 
  \begin{bmatrix}
    2 & -5 & -1 \\ 4 & -4 & -2 \\ -2 & 8 & -4
  \end{bmatrix}= \begin{bmatrix}
    6 & 0 & 0 \\ 0 & 6 & 0 \\ 0 & 0 & 6
  \end{bmatrix}
  $$
\end{frame}


\begin{frame}{Verdadeiro ou Falso}
  \begin{block}{1}
$\begin{pmatrix}
  \cos(\theta) & -\sin(\theta) \\
  \sin(\theta) & \cos{\theta}
\end{pmatrix}$ é invertível para todo $\theta \in \mathbb{R}.$
\end{block}
\end{frame}

\begin{frame}{Verdadeiro ou Falso}
  \begin{block}{2}
Se $T : \mathbb{R}^3 \to \mathbb{R}^3$ é linear e injetora,
e se $A$ é sua matriz associada, então $Ax=b$ têm, no máximo,
uma única solução.
\end{block}
\end{frame}

\begin{frame}{Verdadeiro ou Falso}
  \begin{block}{3}
  Uma aplicação linear $T:\mathbb{R}^3 \to \mathbb{R}^2$ não pode ser sobrejetora.
  \end{block}
\end{frame}


\begin{frame}{Verdadeiro ou Falso}
  \begin{block}{4}
  A aplicação $T(x,y) = (x -2y, \sqrt{2}x)$ é linear
  \end{block}
\end{frame}

\begin{frame}{Verdadeiro ou Falso}
  \begin{block}{5}
  A aplicação $T(x,y) = (2xy, 3x+y)$ é linear
  \end{block}
\end{frame}


\end{document}
