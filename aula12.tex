\documentclass{beamer}
%
% Choose how your presentation looks.
%
% For more themes, color themes and font themes, see:
% http://deic.uab.es/~iblanes/beamer_gallery/index_by_theme.html
%
\mode<presentation>
{
  \usetheme{default}      % or try Darmstadt, Madrid, Warsaw, ...
  \usecolortheme{default} % or try albatross, beaver, crane, ...
  \usefonttheme{default}  % or try serif, structurebold, ...
  \setbeamertemplate{navigation symbols}{}
  \setbeamertemplate{caption}[numbered]
} 

\usepackage[brazil]{babel}
\usepackage[utf8]{inputenc}
\usepackage[T1]{fontenc}

\usepackage{tikz}

\title[Exercícios]{Exercícios}
\author{MAP 2110 - Diurno}
\institute{IME USP}
\date{9 de junho}

\begin{document}



\begin{frame}
  \titlepage
\end{frame}



\begin{frame}{}
  Seria bom ter a imagem do que acontece em dimensão 3
  para entender algumas provas:
  $$ \det(A) = \left( a_{11}
  \begin{vmatrix}
   a_{22} & a_{23} \\
   a_{32} & a_{33}
 \end{vmatrix} -
  a_{21}
  \begin{vmatrix}
   a_{12} & a_{13} \\
   a_{32} & a_{33}
 \end{vmatrix} + a_{31}
 \begin{vmatrix}
   a_{12} & a_{13}\\
   a_{22} & a_{23}
 \end{vmatrix} \right)$$
 onde 
 $$ A = \begin{bmatrix}
  a_{11} & a_{12} & a_{13} \\
  a_{21} & a_{22} & a_{23} \\
  a_{31} & a_{32} & a_{33} 
   \end{bmatrix}$$
\end{frame}

\begin{frame}{Exercício 1 - pg183}
 \begin{gather*}
   \begin{vmatrix}
     a+px & b+qx & c+rx \\
     p+ux & q+vx & r+wx \\
     u+ ax & v+bx & w + cx
   \end{vmatrix}= \\ 
   \textcolor{red}{\begin{vmatrix}
    a & b & c \\
    p+ux & q+vx & r+wx \\
    u+ ax & v+bx & w + cx
  \end{vmatrix}} + \textcolor{blue}{\begin{vmatrix}
    px & qx & rx \\
    p+ux & q+vx & r+wx \\
    u+ ax & v+bx & w + cx
  \end{vmatrix}}
 \end{gather*} 
 
\end{frame}
\begin{frame}
\begin{gather*}
  \textcolor{red}{\begin{vmatrix}
    a & b & c \\
    p+ux & q+vx & r+wx \\
    u+ ax & v+bx & w + cx
  \end{vmatrix}}= \\
  \textcolor{red}{\begin{vmatrix}
    a & b & c \\
    p & q & r \\
    u+ ax & v+bx & w + cx
  \end{vmatrix}} + \textcolor{red}{\begin{vmatrix}
    a & b & c \\
    ux & vx & wx \\
    u+ ax & v+bx & w + cx
  \end{vmatrix}} = \\
  \textcolor{red}{\begin{vmatrix}
    a & b & c \\
    p & q & r \\
    u  & v & w 
  \end{vmatrix}} +\textcolor{red}{\begin{vmatrix}
    a & b & c \\
    p & q & r \\
   ax & bx & cx
  \end{vmatrix}} =
  \textcolor{red}{\begin{vmatrix}
    a & b & c \\
    p & q & r \\
    u & v & w 
  \end{vmatrix}}
\end{gather*}
\end{frame}

\begin{frame}{Parte Azul}
  \begin{gather*}
    \textcolor{blue}{\begin{vmatrix}
      px & qx & rx \\
      p+ux & q+vx & r+wx \\
      u+ ax & v+bx & w + cx
    \end{vmatrix}} = \\
    \textcolor{blue}{x\begin{vmatrix}
      p & q & r \\
      p & q & r \\
      u+ ax & v+bx & w + cx
    \end{vmatrix}} + \textcolor{blue}{x\begin{vmatrix}
      p & q & r \\
      ux & vx & wx \\
      u+ ax & v+bx & w + cx
    \end{vmatrix}} =\\
    \textcolor{blue}{x^2\begin{vmatrix}
      p & q & r \\
      u & v & w \\
      u  & v & w 
    \end{vmatrix}} + \textcolor{blue}{x^2\begin{vmatrix}
      p & q & r \\
      u & v & w \\
      ax & bx & cx
    \end{vmatrix}}=\textcolor{blue}{x^3\begin{vmatrix}
      p & q & r \\
      u & v & w \\
      a & b & c
    \end{vmatrix}}
  \end{gather*}
\end{frame}

\begin{frame}{Juntando as partes}
  \begin{gather*}
    \begin{vmatrix}
      a+px & b+qx & c+rx \\
      p+ux & q+vx & r+wx \\
      u+ ax & v+bx & w + cx
    \end{vmatrix}= (1+x^3)\begin{vmatrix}
      a & b & c \\
      p & q & r \\
      u & v & w 
    \end{vmatrix}
  \end{gather*} 
\end{frame}
 
\begin{frame}{Exercício 4 pg 183}
  \begin{gather*}
    \begin{vmatrix}
      1 & a & a^3 \\
      1 & b & b^3 \\
      1 & c & c^3
    \end{vmatrix} = \begin{vmatrix}
      1 & a & a^3 \\
      0& b-a & b^3-a^3 \\
      0 & c-a & c^3-a^3
    \end{vmatrix} = \begin{vmatrix}
      1 & a & 0 \\
      0 & b-a & b^3 -ba^2 \\
      0 & c-a & c^3 -ca^2
    \end{vmatrix} = \\
   \begin{vmatrix}
    1 & 0 & 0 \\
    0 & b-a & b^3 -ba^2 \\
    0 & c-a & c^3 -ca^2
  \end{vmatrix} = (b-a)(c-a) \begin{vmatrix}
    1 & 0& 0 \\
    0 & 1 & b(b+a) \\
    0 & 1 & c(c+a)
  \end{vmatrix}
  \end{gather*}
  Na segunda igualdade subtraimos $a^2$ vezes a segunda coluna da terceira coluna.
  e na terceira fizemos a segunda coluna menos $a$ vezes a primeira.
\end{frame}
\begin{frame}
 \begin{gather*} \begin{vmatrix}
    1 & a & a^3 \\
    1 & b & b^3 \\
    1 & c & c^3
  \end{vmatrix} = (b-a)(c-a)(c(c+a)-b(b+a) \textcolor{green}{+cb -cb})= \\
  (b-a)(c-a)(c-b)(a+b+c)
\end{gather*}
\end{frame}

\begin{frame}{Exercício 5, pg183}
\begin{gather*}
  \begin{vmatrix}
    3R_1 + 2 R_3 \\
    2R_1 +5 R_2
  \end{vmatrix} = \begin{vmatrix}
    3R_1 + 2R_3 \\ 
    2R_1
  \end{vmatrix} + \begin{vmatrix}
    3R_1 + 2 R_3 \\
    5R_2
  \end{vmatrix} = \\
 4 \begin{vmatrix}
    R_3 \\ R_1
  \end{vmatrix} + 15\begin{vmatrix}
    R1 \\ R_2
  \end{vmatrix} + 10 \begin{vmatrix}
    R_3 \\ R_2
  \end{vmatrix}
\end{gather*}
\end{frame}
\begin{frame}{Se $R_3 = R_2$}
  \begin{gather*}
    \begin{vmatrix}
      3R_1 + 2 R_2 \\
      2R_1 +5 R_2
    \end{vmatrix} = \begin{vmatrix}
      3R_1^T + 2 R_2^T &
      2R_1^T +5 R_2^T
    \end{vmatrix} = \\
    \det(\begin{pmatrix}
      R_1^T & R_2^T
    \end{pmatrix}\begin{pmatrix}
      3 & 2 \\ 2 & 5
    \end{pmatrix}) = \begin{vmatrix}
      R_1 \\ R_2
    \end{vmatrix}\begin{vmatrix}
      3 & 2 \\ 2 & 5
    \end{vmatrix} = 55
  \end{gather*}
\end{frame}

\begin{frame}{Exercício 5 pg 148}
  $A$ é $3\times 3$ e
  \begin{gather*}
    \det(2A^{-1})=-4 = \det(A^3.{B^{-1}}^T) \implies \\
    2^3\det(A^{-1}) = 8/\det(A)=-4 \implies \det(A) = -2 \\
    -4 = \det(A^3)\det(B^{-1}) = \det(A)^3\det(B)^{-1} \implies \\
    \det(B) = \frac{1}{-4}\det(A)^3 = 2
  \end{gather*}
\end{frame}

\begin{frame}{Exercício 8, pg 148}
  a)
\begin{gather*}
  \begin{bmatrix}
    2 & 1 \\ 3 & 7
  \end{bmatrix}\begin{bmatrix}
    x \\ y
  \end{bmatrix}=\begin{bmatrix}
    1 \\ -2
  \end{bmatrix} \\
x={ \begin{vmatrix}
  1 & 1 \\
  -2 & 7
\end{vmatrix} \over \begin{vmatrix}
  2 & 1 \\ 3 & 7
\end{vmatrix}} \text{ e }y={ \begin{vmatrix}
  2 & 1 \\
  3 & -2
\end{vmatrix} \over \begin{vmatrix}
  2 & 1 \\ 3 & 7
\end{vmatrix}}
x=\frac{9}{11} \text{ e } y = \frac{-7}{11}
\end{gather*}
\end{frame}

\begin{frame}
  \begin{gather*}
    \begin{bmatrix}
      4 & -1 & 3 \\
      6 & 2 & -1 \\
      3 & 3 & 2
    \end{bmatrix}\begin{bmatrix}
      x \\ y \\ z
    \end{bmatrix} = \begin{bmatrix}
      1 \\ 0 \\ -1
    \end{bmatrix} \\
    x = { \begin{vmatrix}
      1 & -1 & 3 \\
      0 & 2 & -1 \\
      -1 & 3 &2
     \end{vmatrix} \over \begin{vmatrix}
       4 & -1 & 3 \\
       6 & 2 & -1 \\
       3 & 3 & 2
     \end{vmatrix}} \text{ e }
    y = { \begin{vmatrix}
     4& 1 &  3 \\
     6 & 0 &  -1 \\
      3&-1 & 2
     \end{vmatrix} \over \begin{vmatrix}
       4 & -1 & 3 \\
       6 & 2 & -1 \\
       3 & 3 & 2
     \end{vmatrix}} \text{ e }
     z = { \begin{vmatrix}
      4 & -1 & 1 \\
      6 & 2 & 0 \\
      3 & 3 &-1
     \end{vmatrix} \over \begin{vmatrix}
       4 & -1 & 3 \\
       6 & 2 & -1 \\
       3 & 3 & 2
     \end{vmatrix}} \\
     x=\frac{12}{79}\text{ }y=\frac{-37}{79}\text{ }z=\frac{-2}{79}
\end{gather*}
\end{frame}

\end{document}
