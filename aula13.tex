\documentclass{beamer}
%
% Choose how your presentation looks.
%
% For more themes, color themes and font themes, see:
% http://deic.uab.es/~iblanes/beamer_gallery/index_by_theme.html
%
\mode<presentation>
{
  \usetheme{default}      % or try Darmstadt, Madrid, Warsaw, ...
  \usecolortheme{default} % or try albatross, beaver, crane, ...
  \usefonttheme{default}  % or try serif, structurebold, ...
  \setbeamertemplate{navigation symbols}{}
  \setbeamertemplate{caption}[numbered]
} 

\usepackage[brazil]{babel}
\usepackage[utf8]{inputenc}
\usepackage[T1]{fontenc}

\usepackage{tikz}

\title[diagonalização]{Diagonalização}
\author{MAP 2110 - Diurno}
\institute{IME USP}
\date{16 de junho}

\begin{document}



\begin{frame}
  \titlepage
\end{frame}



\begin{frame}{Exemplo: Equação de recorrência}
  As relações de recorrência são formas de descrever a evolução de uma variável conforme avançamos no
  "tempo". No nosso caso o "tempo" é o conjunto de números naturais $\mathbb{N}$ ou inteiros $\mathbb{Z}$.
  Vamos analisar um exemplo simples. Tomei um empréstimo de $\text{R}\$1000,00$ com juros de $6\%$ ao ano.
  Quanto devo pagar ao fim do quinto anos (O juros serão cobrados uma vez ao ano!)
  \begin{itemize}
    \item $P_k = 1.06*P_{k-1}$  relação de recorrência.
    \item $P_k = (1.06)^k P_0$ solução (permite resolver o problema)
    \item $ P_5 = (1.06)^5*(1000) = \text{R}\$1338.23$
  \end{itemize}
\end{frame}

\begin{frame}{}
  \begin{gather*}
    I_k = 2I_{k-1}\\
    I_{10} = 2^{10}I_0 = 1024I_0
  \end{gather*}
  \begin{gather*}
    I_k = 2I_{k-1} -2D_{k-1} \\
    D_k = 0.01*I_{k-1} \\
    \begin{bmatrix}
      I_k \\ D_k
    \end{bmatrix} = \begin{pmatrix}
      2 & -1 \\ 0.01 & 0
    \end{pmatrix}\begin{bmatrix}
      I_{k-1} \\ D_{k-1}
    \end{bmatrix} \\
    \begin{bmatrix}
      I_{10} \\ D_{10}
    \end{bmatrix} = \begin{pmatrix}
      2 & -1 \\ 0.01 & 0
    \end{pmatrix}^{10}\begin{bmatrix}
      I_{0} \\ D_{0}
    \end{bmatrix}
\end{gather*}

\end{frame}
\begin{frame}{Sequência de Fibonacci}
  \begin{gather*}
    F_0=1\text{ }F_1=1 \\
    F_{k+1}=F_k + F_{k-1} \text{ recorrência de ordem 2} 
  \end{gather*}
  Podemos escrever a equação de Fibonacci em um sistema:
  \begin{gather*}
    F_{k+1}=F_k + F_{k-1} \\
    F_k = F_k \text{ ou } \\
    \begin{bmatrix}
      F_{k+1} \\
      F_k
    \end{bmatrix}= \begin{bmatrix}
      1 & 1 \\ 1 & 0
    \end{bmatrix}\begin{bmatrix}
      F_{k} \\ F_{k-1}
    \end{bmatrix} \\
    \begin{bmatrix}
      F_{10} \\
      F_{9}
    \end{bmatrix}= \begin{bmatrix}
      1 & 1 \\ 1 &0
    \end{bmatrix}^9 \begin{bmatrix}
      1 \\ 1
    \end{bmatrix}
  \end{gather*}
  
\end{frame}

\begin{frame}{Exemplo do livro}
  \begin{gather*}
    a_{k+1} = \frac{a_k}{2} + \frac{j_k}{4} \\
    j_{k+1} = 2a_k
  \end{gather*}
  \begin{gather*}
    \begin{bmatrix}
      a_{k+1} \\
      j_{k+1}
    \end{bmatrix} = \begin{bmatrix}
      \frac{1}{2} & \frac{1}{4} \\
      2 & 0
    \end{bmatrix}\begin{bmatrix}
      a_k \\ b_k
    \end{bmatrix} \text{ com } \begin{bmatrix}
      a_0 \\ j_0 
    \end{bmatrix}= \begin{bmatrix}
      400 \\ 40
    \end{bmatrix}
  \end{gather*}
  
\end{frame}

\begin{frame}{Problema geral}

  \begin{gather*}
    \mathbf{v}_{k+1} = A \mathbf{v}_k \\
    \text{ para achar a solução em tempo n} \\
    \mathbf{v}_n = A^n \mathbf{v}_0
\end{gather*}
Como calcular $A^n$ ?
  
\end{frame} 

\begin{frame}
  Uma matriz quadrada $D =[d_{ij}]$ é uma matriz diagonal 
  quando $d_{ij}=0$ para $i\neq j.$ Neste caso é mais 
  fácil achar a pontência $D^n$, pois esta também será
  diagonal, bastando calcular as potências dos números 
  na diagonal.
  \textbf{exemplo}
  
  \begin{gather*}
    A = \begin{bmatrix}
      2 & 0 \\ 0 & 1
    \end{bmatrix} \text{ então } A^{10} = \begin{bmatrix}
      1024 & 0 \\ 0 & 1
    \end{bmatrix}
  \end{gather*}
\end{frame}

\begin{frame}{Tentando facilitar as contas}
  Considere o problema geral
  $$ \mathbf{v}_{k+1} = A \mathbf{v}_k $$ e nossa intensão
  é achar uma fórmula para  $\mathbf{v}_n$ em função de 
  $\mathbf{v}_0$. Vamos escolher uma matriz invertível 
  $P$, com as dimensões de $A$ e fazer a mudança de variável 

  $$ \mathbf{v_k} = P\mathbf{u_k} \text{  ou  } P^{-1}\mathbf{v_k} = \mathbf{u_k} $$
  Com essa modificação nossa fórmula de recorrência fica:
  \begin{gather*}
    P\mathbf{u_{k+1}}=AP\mathbf{u}_k  \implies\\
    \mathbf{u_{k+1}} = P^{-1}AP\mathbf{u_k} \\
    \mathbf{u_{k+1}} = B\mathbf{u_k}  \text{ com } B = P^{-1}AP\\
  \end{gather*}
  
\end{frame}
\begin{frame}{Será que $B$ pode ser diagonal?}
  Se $B$ for diagonal então teríamos um jeito de calcular $A^n$ a partir de $B^n$ pois:
  \begin{gather*}
    B^n = (P^{-1}AP)^n=P^{-1}APP^{-1}AP\cdots P^{-1}AP = P^{-1}A^nP \implies \\
    A^n = PB^nP^{-1}
  \end{gather*}
  \textbf{Exemplo:}
  \begin{gather*}
    \begin{bmatrix}
      -1 & 3 \\
      0 & 2
    \end{bmatrix} = \begin{bmatrix}
      1 & 1 \\ 1 & 0 
    \end{bmatrix}\begin{bmatrix}
      2 & 0 \\ 0 & -1
    \end{bmatrix}\begin{bmatrix}
      0 & 1 \\ 1 & -1
    \end{bmatrix} \\
    \begin{bmatrix}
      -1 & 3 \\
      0 & 2
    \end{bmatrix}^5 = \begin{bmatrix}
      1 & 1 \\ 1 & 0 
    \end{bmatrix}\begin{bmatrix}
      2^5 & 0 \\ 0 & (-1)^5
    \end{bmatrix}\begin{bmatrix}
      0 & 1 \\ 1 & -1
    \end{bmatrix} = \begin{bmatrix}
      -1 & 33 \\ 0 & 32
    \end{bmatrix}
  \end{gather*}
  
\end{frame}

\begin{frame}{Problema}
  Será que para toda matriz quadrada $A$ existe uma 
  matriz $P$ invertível talque $P^{-1}AP=B$ seja uma 
  matriz diagonal?
  
  Vamos tentar com uma matriz $2\times 2$

\textbf{Experimento:}
 suponha que $A= \begin{pmatrix}
   1 & 1 \\ 1 & 0
 \end{pmatrix}$ e que existe uma matriz $P=\begin{bmatrix}
   p_1 & q_1 \\ p_2 & q_2
 \end{bmatrix}$ que seja invertível e que 
 $P^{-1}AP = \begin{bmatrix}
   \lambda_1 & 0 \\ 0 & \lambda_2
 \end{bmatrix}$ Então, multiplicando à esquerda por $P$ temos:
 $$ A\begin{bmatrix}
  p_1 & q_1 \\ p_2 & q_2
\end{bmatrix} = \begin{bmatrix}
  p_1 & q_1 \\ p_2 & q_2
\end{bmatrix}\begin{bmatrix}
  \lambda_1 & 0 \\ 0 & \lambda_2
\end{bmatrix}$$
\end{frame}

\begin{frame}
  Vamos agora igualar as colunas das matrizes dos dois lados da equação:
  Para a primeira coluna temos.
  $$ A\begin{bmatrix}
    p_1 \\ p_2
  \end{bmatrix} = \begin{bmatrix}
    p_1 & q_1 \\ p_2 & q_2
  \end{bmatrix}\begin{bmatrix}
    \lambda_1 \\ 0
  \end{bmatrix} = \lambda_1 \begin{bmatrix}
    p_1 \\ p_2
  \end{bmatrix} $$
  e similarmente para a segunda coluna:
  $$ A \begin{bmatrix}
    q_1 \\ q_2
  \end{bmatrix} = \begin{bmatrix}
    p_1 & q_1 \\ p_2 & q_2
  \end{bmatrix}\begin{bmatrix}
    0 \\ \lambda_2
  \end{bmatrix} = \lambda_2 \begin{bmatrix}
    q_1 \\ q_2
  \end{bmatrix} $$ Estas equações também podem ser 
  escritas como:
  $$ (A - \lambda_1I) \begin{bmatrix}
    p_1 \\ p_2
  \end{bmatrix} = 0 \text{ e } (A - \lambda_2I) \begin{bmatrix}
    q_1 \\ q_2
  \end{bmatrix} = 0$$
\end{frame}

\begin{frame}
  Para $P$ ser invertível as colunas não podem ser nulas, então
  devemos ter soluções não triviais destas equações homogêneas.
  ou seja $\det(\lambda_1I -A) = 0$ e $\det(\lambda_2I-A)=0$
  Isso vai restringir as possibilidades para $\lambda_1$ e $\lambda_2$.
  Calculando o $\det(\lambda I -A)$ temos
  \begin{gather*}
    \det(\lambda I -A) = \begin{vmatrix}
      \lambda - 1 & -1 \\
      -1 & \lambda
    \end{vmatrix} = \lambda^2 -\lambda -1 = p(\lambda)
  \end{gather*}
  As raízes deste polinômio serão os nossos $\lambda_1$ e $\lambda_2$. Chamaremos $p(\lambda)$ de polinômio característico
  de $A$ e suas raízes (que poderiam ser números complexos), chamaremos de autovalores de $A$.
  No nosso caso experimental temos duas raízes
  $$\lambda_1=\frac{1+\sqrt{5}}{2} \text{ e } \lambda_2=\frac{1-\sqrt{5}}{2}$$
  Agora precisamos achar as colunas $\begin{bmatrix}
    p_1 \\ p_2
  \end{bmatrix}$ e $\begin{bmatrix}
    q_1 \\ q_2
  \end{bmatrix}$

\end{frame}
\begin{frame}{encontrando os autovetores}

  Para $\lambda_1$ temos
  \begin{gather*}
    p_1 + p_2 = \lambda_1 p_1 \\
    p_1 = \lambda_1 P_2 \\
    \text{ soluções são da forma } \\
    p\begin{bmatrix}
      \lambda_1 \\ 1
    \end{bmatrix}=\lambda\begin{bmatrix}
      \frac{1+\sqrt{5}}{2} \\ 1
    \end{bmatrix}
  \end{gather*}
  A outra equação é similar e temos
  \begin{gather}
    \begin{bmatrix}
      q_1 \\ q_2
    \end{bmatrix} = \lambda\begin{bmatrix}
    \frac{1 -\sqrt{5}}{2} \\ 1
    \end{bmatrix} \\ 
    P = \begin{bmatrix}
      \frac{1+\sqrt{5}}{2} & \frac{1-\sqrt{5}}{2}\\ 1 & 1
    \end{bmatrix} \text{ e } P^{-1} =\frac{1}{\sqrt{5}}\begin{bmatrix}
      1 & \frac{-1+\sqrt{5}}{2} \\
      -1 & \frac{1+\sqrt{5}}{2}
    \end{bmatrix} 
  \end{gather}
\end{frame}
\begin{frame}{Sumário}
  Se $A$ é uma matriz quadrada $n \times n$
  \begin{itemize}
    \item $p(\lambda) = \det(\lambda I - A)$ é um polinômio de grau $n$
    \item Suas raízes serão chamadas autovalores de $A$
    \item Um vetor não nulo $\mathbf{v}$ que satisfaz $A\mathbf{v}-\lambda\mathbf{v}=0$ é chamado de autovetor associado a $\lambda$.
    \item Se encontramos $n$ autovetores linearmente independentes $\mathbf{v}_1\dots \mathbf{v}_n$, então a matriz $P=[\mathbf{v_1}\cdots\mathbf{v}_n]$ é invertível e $P^{-1}AP$ é diagonal.
    \item No caso acima dizemos que $A$ é diagonalizável.
  \end{itemize}
  
\end{frame}
\begin{frame}
  $$A = \begin{bmatrix}
    -6 & 5 \\ 0 & 4
  \end{bmatrix}$$
  \begin{enumerate}
    \item Achar o polinômio característico e os auto valores.
    \item Achar os autovetores e a matriz $P$ inversível tal que $P^{-1}AP$ seja diagonal.
  \end{enumerate}
\end{frame}
\end{document}
