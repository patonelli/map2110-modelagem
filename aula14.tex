\documentclass{beamer}
%
% Choose how your presentation looks.
%
% For more themes, color themes and font themes, see:
% http://deic.uab.es/~iblanes/beamer_gallery/index_by_theme.html
%
\mode<presentation>
{
  \usetheme{default}      % or try Darmstadt, Madrid, Warsaw, ...
  \usecolortheme{default} % or try albatross, beaver, crane, ...
  \usefonttheme{default}  % or try serif, structurebold, ...
  \setbeamertemplate{navigation symbols}{}
  \setbeamertemplate{caption}[numbered]
} 

\usepackage[brazil]{babel}
\usepackage[utf8]{inputenc}
\usepackage[T1]{fontenc}

\usepackage{tikz}

\title[diagonalização]{Exercícios}
\author{MAP 2110 - Diurno}
\institute{IME USP}
\date{23 de junho}

\begin{document}



\begin{frame}
  \titlepage
\end{frame}

\begin{frame}{Sumário}
  Se $A$ é uma matriz quadrada $n \times n$
  \begin{itemize}
    \item $p(\lambda) = \det(\lambda I - A)$ é um polinômio de grau $n$
    \item Suas raízes serão chamadas autovalores de $A$
    \item Um vetor não nulo $\mathbf{v}$ que satisfaz $A\mathbf{v}-\lambda\mathbf{v}=0$ é chamado de autovetor associado a $\lambda$.
    \item Se encontramos $n$ autovetores linearmente independentes $\mathbf{v}_1\dots \mathbf{v}_n$, então a matriz $P=[\mathbf{v_1}\cdots\mathbf{v}_n]$ é invertível e $P^{-1}AP$ é diagonal.
    \item No caso acima dizemos que $A$ é diagonalizável.
  \end{itemize}
  
\end{frame}
\begin{frame}{1}
  $$A = \begin{bmatrix}
    -6 & 5 \\ 0 & 4
  \end{bmatrix}$$
  \begin{enumerate}
    \item Achar o polinômio característico e os auto valores.
    \item Achar os autovetores e a matriz $P$ inversível tal que $P^{-1}AP$ seja diagonal.
  \end{enumerate}
\end{frame}

\begin{frame}{2}
  Se $p(\lambda)$ é o polinômio característico de uma matriz $A$, e se $P$ é uma matriz invertível, então 
  $P^{-1}AP$ tem o mesmo polinômio característico $p(\lambda)$
\end{frame}

\begin{frame}{3}
  A sequência de Fibonacci é assim:
  \begin{gather}
    F_0=0 \text{ e } F_1 = 1 \\
    F_{k+1} = F_k + F_{k-1}
  \end{gather}
  Mostre que vale :
  \begin{gather*}
    \begin{bmatrix}
      F_{k+1} & F_{k} \\
      F_k & F_{k-1}
    \end{bmatrix} = \begin{bmatrix}
      1 & 1 \\
      1 & 0
    \end{bmatrix}^k \\
    \text{E a fórmula de Cassini: }\\
    F_{n+1}F_{n-1} - {F_n}^2 =(-1)^n
  \end{gather*}
  
\end{frame}

\begin{frame}{4}
  Dê exemplo de uma matriz que tenha os seguintes polinômios característicos:
  \begin{enumerate}
    \item $p(\lambda) = \lambda^2 + \lambda -2$
    \item $p(\lambda) = \lambda^2 + 1$
    \item $p(\lambda) = \lambda^2 + b\lambda + c$
    \item $p(\lambda) = 2\lambda^2 -2\lambda -2$
  \end{enumerate}
  
\end{frame}

\begin{frame}{5}
  \begin{gather*}
    A= \begin{bmatrix}
      1 & 0 \\ 0 & 1
    \end{bmatrix} \text{ e } B= \begin{bmatrix}
      1 & 1 \\ 0 & 1 
    \end{bmatrix}
  \end{gather*}
  têm o mesmo polinômio característico ( e portanto os mesmos autovalores), mas não tem os mesmos autovetores!
\end{frame}

\begin{frame}{6}
  \begin{gather*}
    \mathbf{v}_1 = \begin{bmatrix}
      -1 \\ 2
    \end{bmatrix}\text{ }\mathbf{v}_2 = \begin{bmatrix}
      1 \\ 1
    \end{bmatrix} \text{ e } \mathbf{u}=\begin{bmatrix}
      1 \\ 2
    \end{bmatrix}
  \end{gather*}
  Se $\mathbf{v}_1$ é um autovetor associado ao autovalor $\lambda_1=1$ de uma matriz $A$ e $\mathbf{v}_2$ é outro
  autovetor de $A$, será que podemos achar a matriz original $A$ sabendo que 
  $$ A\mathbf{u}=\begin{bmatrix} \frac{11}{3}\\
    \frac{14}{3}
    \end{bmatrix} ?$$
\end{frame}
\end{document}
