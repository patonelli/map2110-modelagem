\documentclass{beamer}
%
% Choose how your presentation looks.
%
% For more themes, color themes and font themes, see:
% http://deic.uab.es/~iblanes/beamer_gallery/index_by_theme.html
%
\mode<presentation>
{
  \usetheme{default}      % or try Darmstadt, Madrid, Warsaw, ...
  \usecolortheme{default} % or try albatross, beaver, crane, ...
  \usefonttheme{default}  % or try serif, structurebold, ...
  \setbeamertemplate{navigation symbols}{}
  \setbeamertemplate{caption}[numbered]
} 

\usepackage[brazil]{babel}
\usepackage[utf8]{inputenc}
\usepackage[T1]{fontenc}

\usepackage{tikz}
\usepackage{siunitx}

\title[diagonalização]{Sistemas Dinâmicos Lineares}
\author{MAP 2110 - Diurno}
\institute{IME USP}
\date{30 de junho}

\begin{document}



\begin{frame}
  \titlepage
\end{frame}
\begin{frame}{Quando o polinômio característico tem raízes duplas}
  \begin{gather*}
    A= \begin{bmatrix}
      1 & 0 \\ 0 & 1
    \end{bmatrix} \text{ e } B= \begin{bmatrix}
      1 & 1 \\ 0 & 1 
    \end{bmatrix}
  \end{gather*}
  têm o mesmo polinômio característico $p(\lambda) = \lambda^2 -2\lambda +1$ ( e portanto os mesmos autovalores), mas não tem os mesmos autovetores!
  A segunda matriz tem apenas uma direção de autovetores.
\end{frame}

\begin{frame}{Recuperando informações com testes}
  \begin{gather*}
    \mathbf{v}_1 = \begin{bmatrix}
      -1 \\ 2
    \end{bmatrix}\text{ }\mathbf{v}_2 = \begin{bmatrix}
      1 \\ 1
    \end{bmatrix} \text{ e } \mathbf{u}=\begin{bmatrix}
      1 \\ 2
    \end{bmatrix}
  \end{gather*}
  Se $\mathbf{v}_1$ é um autovetor associado ao autovalor $\lambda_1=1$ de uma matriz $A$ e $\mathbf{v}_2$ é outro
  autovetor de $A$, será que podemos achar a matriz original $A$ sabendo que 
  $$ A\mathbf{u}=\begin{bmatrix} \frac{11}{3}\\
    \frac{14}{3}
    \end{bmatrix} ?$$
\end{frame}

\begin{frame}
  \begin{gather*}
    A\mathbf{v}_1 = \mathbf{v}_1 \\
    A\mathbf{v}_2 = \lambda_2\mathbf{v}_2\\
    P= \begin{bmatrix}
      -1 & 1 \\ 2 & 1
    \end{bmatrix}\text{ e } P^{-1}=\frac{1}{3}\begin{bmatrix}
      -1 & 1 \\ 2 & 1
    \end{bmatrix}  \\
    P^{-1}AP=\begin{bmatrix}
      1 & 0 \\ 0 & \lambda_2
    \end{bmatrix}\\
    A\mathbf{u}=
    \frac{1}{3} \begin{bmatrix}
      -1 & 1 \\ 2 & 1
    \end{bmatrix} \begin{bmatrix}
      1 & 0 \\ 0 & \lambda_2
    \end{bmatrix} \begin{bmatrix}
      -1 & 1 \\ 2 & 1
    \end{bmatrix}\begin{bmatrix}
      1 \\ 2
    \end{bmatrix} = \frac{1}{3}\begin{bmatrix}
      11 \\ 14
    \end{bmatrix} \\
    \lambda_2=3
  \end{gather*}
\end{frame}

\begin{frame}{trajetória de um sistema}
  \begin{gather*}
    \mathbf{x}_{k+1} =A\mathbf{x}_k \\
    \mathbf{x}_0 = \mathbf{v}
  \end{gather*}
  Então a solução deste sistema é $\mathbf{x}_k = A^k\mathbf{v}$ e a 
  sequência de vetores $\{\mathbf{v}_k\}$ com $v_k = A^k\mathbf{v}$ dizemos 
  que é a trajetória que passa por $\mathbf{v}$. De uma forma geral
  a teoria dos sistemas dinâmicos procura saber o que acontece
  com as trajetórias do sistema.
  
\end{frame}

\begin{frame}{Sistemas dinâmicos e autovalores}
  No caso linear, o que acontece com as trajetórias depende
  dos autovalores da matriz $A$.
  \textbf{Exemplo}
  \begin{gather*}
    \begin{bmatrix}
      x_{k+1} \\ y_{k+1} 
    \end{bmatrix} = \begin{bmatrix}
      3 & 0 \\ 0 & 2
    \end{bmatrix}\begin{bmatrix}
      x_k \\ y_k
    \end{bmatrix}
  \end{gather*}
  que tem a solução
  \begin{gather*}
    \begin{bmatrix}
      x_k \\ y_k 
    \end{bmatrix} =\begin{bmatrix}
      3^kx_0 \\ 2^k y_0
    \end{bmatrix} = 3^kx_0\textcolor{blue}{\begin{bmatrix}
      1 \\ 0
    \end{bmatrix}} + 2^k y_0 \textcolor{green}{\begin{bmatrix}
      0 \\ 1
    \end{bmatrix}}
  \end{gather*}
  
\end{frame}

\begin{frame}
  \begin{itemize}
  \item Temos, neste caso, dois autovalores reais, com autovalores 
  maiores que $1$, 

  \item como consequência, as trajetórias por 
  qualquer ponto fora da origem se afastarão cada vez mais da origem.

  \item O autovalor $3$ é maior do que $2$ (em valor absoluto),
  então dizemos que é um autovalor dominante e as trajetórias
  se acumularão na direção do autovetor associado a este autovalor.

  \item A origem $(0,0)$ é um ponto de equilíbrio instável
\end{itemize}
\end{frame}

\begin{frame}{Exemplo 2}
  \begin{gather*}
    \begin{bmatrix}
      x_{k+1} \\ y_{k+1} 
    \end{bmatrix} = \begin{bmatrix}
      1 & 0 \\ 0 & 0.5
    \end{bmatrix}\begin{bmatrix}
      x_k \\ y_k
    \end{bmatrix}
  \end{gather*}
  \begin{gather*}
  \begin{bmatrix}
    x_k \\ y_k 
  \end{bmatrix} =\begin{bmatrix}
    x_0 \\ (0.5)^k y_0
  \end{bmatrix} = x_0\textcolor{blue}{\begin{bmatrix}
    1 \\ 0
  \end{bmatrix}} + (0.5)^k y_0 \textcolor{green}{\begin{bmatrix}
    0 \\ 1
  \end{bmatrix}}
\end{gather*}
\end{frame}

\begin{frame}
  \begin{itemize}
    \item Um autovalor é $1$ e o outro tem valor absoluto menor que $1$.
    \item Como $(0.5)^k$ tende a zero a trajetória se aproxima de um equilíbrio.
    \item $x_0\begin{bmatrix}
      1 \\ 0
    \end{bmatrix}$ é um equilíbrio para todo $x_0$.
  \end{itemize}
\end{frame}
\begin{frame}{Exemplo 3}
  \begin{gather*}
    \begin{bmatrix}
      x_{k+1} \\ y_{k+1} 
    \end{bmatrix} = \begin{bmatrix}
      0.4 & 0 \\ 0 & 0.3
    \end{bmatrix}\begin{bmatrix}
      x_k \\ y_k
    \end{bmatrix}
  \end{gather*}
  \begin{gather*}
  \begin{bmatrix}
    x_k \\ y_k 
  \end{bmatrix} =\begin{bmatrix}
    (0.4)^kx_0 \\ (0.3)^k y_0
  \end{bmatrix} = (0.4)^k x_0\textcolor{blue}{\begin{bmatrix}
    1 \\ 0
  \end{bmatrix}} + (0.3)^k y_0 \textcolor{green}{\begin{bmatrix}
    0 \\ 1
  \end{bmatrix}}
\end{gather*}
  
\end{frame}
\begin{frame}
  \begin{itemize}
    \item Dois autovalores com valor absoluto menor que $1$
    \item A trajetória se aproxima da origem sempre.
    \item A origem é um ponto de equilíbrio estável.
  \end{itemize}
\end{frame}

\begin{frame}{Caso de uma matriz diagonalizável}
  Vamos supor que uma matriz $2 \times 2$ seja diagonalizável com autovalores $\lambda_1$ e
  $\lambda_2$ com autovetores associados:
  $$ \mathbf{v}_1 = \begin{bmatrix}
    p_1 \\ p_2
  \end{bmatrix} \text{ e } \mathbf{v}_2=\begin{bmatrix}
    q_1 \\ q_2
  \end{bmatrix}$$

  queremos saber como são as trajetórias de 
  $$ \begin{bmatrix}
    x_{k+1} \\ y_{k+1}\end{bmatrix} = A \begin{bmatrix}
      x_k \\ y_k
  \end{bmatrix}$$
  
\end{frame}


\begin{frame}
  \begin{gather*}
    \begin{bmatrix}
      x_k \\ y_k
    \end{bmatrix} = A^k \begin{bmatrix}
      x_0 \\ y_0
    \end{bmatrix} = \begin{bmatrix}
      p_1 & q_1 \\
      p_2 & q_2 
    \end{bmatrix}\begin{bmatrix}
      \lambda_1^k & 0 \\
      0 & \lambda_2^k
    \end{bmatrix} P^{-1}\begin{bmatrix}
      x_0 \\ y_0
    \end{bmatrix} \\
    \begin{bmatrix}
      x_k \\ y_k
    \end{bmatrix} =\begin{bmatrix}
      \mathbf{v}_1 & \mathbf{v}_2
    \end{bmatrix} \begin{bmatrix}
      \lambda_1^k a \\ \lambda_2^k b
    \end{bmatrix} = \lambda_1^k a\textcolor{blue}{\mathbf{v}_1} + \lambda_2^kb \textcolor{green}{\mathbf{v}_2}
  \end{gather*}

  E agora podemos repetir as análises anteriores  substituindo os
  vetores canônicos $\mathbf{e}_1$ e $\mathbf{e}_2$ pelos autovetores
  $\mathbf{v}_1$ e $\mathbf{v}_2$.
\end{frame}

\begin{frame}{Exemplo}
  \begin{gather*}
    \begin{bmatrix}
      x_{k+1} \\ y_{k+1} 
    \end{bmatrix} = \begin{bmatrix}
      4& -1 \\ 2 & 1
    \end{bmatrix}\begin{bmatrix}
      x_k \\ y_k
    \end{bmatrix}
  \end{gather*}
  Tem como autovalores $3$ e $2$ e como autovetores associados 
  $\mathbf{v}_1=\begin{bmatrix}
    1 \\ 1
  \end{bmatrix}$ e $\mathbf{v}_2=\begin{bmatrix}
    1 \\ 2
  \end{bmatrix}.$ As trajetórias crescem com fator $3$ na direção do 
  $\mathbf{v}_1$ e com fator $2$ na direção do $\mathbf{v}_2$ e para 
  $k \to \infty$ tende para a direção de $\mathbf{v}_1$
\end{frame}

\begin{frame}{Autovalores complexos}
  \begin{gather*}
    \begin{bmatrix}
      x_{k+1} \\ y_{k+1} 
    \end{bmatrix} = \begin{bmatrix}
      0& -1\\ 1 & 0
    \end{bmatrix}\begin{bmatrix}
      x_k \\ y_k
    \end{bmatrix}
  \end{gather*}
  A matriz deste sistema tem o polinômio característico 
  $p(\lambda)= \lambda^2 + 1$ e os autovalores são os números
  complexos $\pm i$
  
\end{frame}

\begin{frame}
  Neste caso sabemos que $A=\begin{bmatrix}
    0 & -1 \\ 1 & 0
  \end{bmatrix}$ é uma matriz de rotação de \ang{90}
  Então a trajetória por qualquer ponto $\begin{bmatrix}
    x \\ y
  \end{bmatrix}$ tem só quatro pontos
  $$\left\{  \begin{bmatrix}
    x \\ y
  \end{bmatrix}, \begin{bmatrix}
    -y \\ x
  \end{bmatrix}, \begin{bmatrix}
    -x \\ -y 
  \end{bmatrix} , \begin{bmatrix}
    y \\ -x
  \end{bmatrix}\right\}$$
\end{frame}

\begin{frame}{Matriz de rotação}
  $$ A= \begin{bmatrix}
    \cos(\theta) & -\sin(\theta) \\
    \sin(\theta)& \cos(\theta)
  \end{bmatrix}$$ é a matriz de rotação de um ângulo
  $\theta$ no sentido anti-horário. Os autovalores desta
  matriz são $\cos(\theta) \pm i\sin(\theta)$.
  Note que
  $$ \begin{bmatrix}
    \cos(\theta) & -\sin(\theta) \\
    \sin(\theta)& \cos(\theta)
  \end{bmatrix}^n=\begin{bmatrix}
    \cos(n\theta) & -\sin(n\theta) \\
    \sin(n\theta)& \cos(n\theta)
  \end{bmatrix} $$
  Então 
  $$ \begin{bmatrix}
    x_k \\ y_k
  \end{bmatrix} = \begin{bmatrix}
    \cos(n\theta) & -\sin(n\theta) \\
    \sin(n\theta)& \cos(n\theta)
  \end{bmatrix}\begin{bmatrix}
    x_0 \\ y_0
  \end{bmatrix} $$
      
\end{frame}

\begin{frame}{Exemplo}
  \begin{gather*}
    A=\begin{bmatrix}
      3 & -4 \\ 4 & 3
    \end{bmatrix} = 5 \begin{bmatrix}
      \frac{3}{5}&-\frac{4}{5}\\[2mm]
      \frac{4}{5}&\frac{3}{5}
    \end{bmatrix} = 5\begin{bmatrix}
      \cos(\theta) & -\sin(\theta) \\
      \sin(\theta)& \cos(\theta)
    \end{bmatrix} \text{ com } \theta = \arccos(3/5)
  \end{gather*}
  Os autovalores de $A$ são $3 \pm 4i$ e as trajetórias são
  $$ \begin{bmatrix}
    x_n \\ y_n 
  \end{bmatrix} = 5^n\begin{bmatrix}
    \cos(n\theta) & -\sin(n\theta) \\
    \sin(n\theta)& \cos(n\theta)
  \end{bmatrix}$$
  
\end{frame}

\begin{frame}{Exemplo}
  $$ A = \begin{bmatrix}
    11 & - 6\\
    15 & -7
  \end{bmatrix} $$ 
  tem autovalores
  $$ \lambda_{1,2}=2\pm 3i $$
  Podemos achar os "autovetores complexos"
  $$ \mathbf{v}_1 = \begin{bmatrix}
    2 \\ 3
  \end{bmatrix} -i\begin{bmatrix}
    0 \\ 1
  \end{bmatrix} \text{ e }\mathbf{v}_2 = \begin{bmatrix}
    2 \\ 3
  \end{bmatrix} + i\begin{bmatrix}
    0 \\ 1
  \end{bmatrix} $$
  Agora definindo 
  $$ Q= \begin{bmatrix}
    2 & 0 \\
    3 & 1
  \end{bmatrix} \text{ temos } Q^{-1}AQ = \begin{bmatrix}
    2 & -3 \\ 3 & 2
  \end{bmatrix}$$


  
\end{frame}

\end{document}
