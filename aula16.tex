\documentclass{beamer}
%
% Choose how your presentation looks.
%
% For more themes, color themes and font themes, see:
% http://deic.uab.es/~iblanes/beamer_gallery/index_by_theme.html
%
\mode<presentation>
{
  \usetheme{default}      % or try Darmstadt, Madrid, Warsaw, ...
  \usecolortheme{default} % or try albatross, beaver, crane, ...
  \usefonttheme{default}  % or try serif, structurebold, ...
  \setbeamertemplate{navigation symbols}{}
  \setbeamertemplate{caption}[numbered]
} 

\usepackage[brazil]{babel}
\usepackage[utf8]{inputenc}
\usepackage[T1]{fontenc}

\usepackage{tikz}
\usepackage{siunitx}

\title[transformações lineares]{Matrizes e transformações lineares}
\author{MAP 2110 - Diurno}
\institute{IME USP}
\date{7 de julho de 2020}

\begin{document}



\begin{frame}
  \titlepage
\end{frame}

\begin{frame}
  \begin{itemize}
    \item Transformações entre espaços são importantes em Geometria.
    \item As transformações importantes são aquelas que preservam algumas propriedadas da figura transformada.
    \item Em geral classificamos as transformações em grupos.
    \item É importante dar uma fórmula para as transformações, e aí entram as matrizes.
  \end{itemize}
\end{frame}
\begin{frame}{Rotações no plano}
  \begin{center}
    
  
  \begin{tikzpicture}
    \draw[gray] (-0.2,-0.2) grid (7.2,7.2);
    \draw[blue,-latex] (0,0)--(7.5,0);
    \draw[blue,-latex] (0,0)--(0,7.5);
    \draw[red,dashed] (0,0) -- (30:7.5);
    \filldraw[blue!40] (5,0) rectangle (6,1);
    \filldraw[blue] (6,1) circle (0.1);
    \draw[red,thick] (3,0) arc (0:30:3) node[anchor=north,yshift=-0.5cm] {$\theta$};
    %\filldraw[blue!40] (30:5) rectangle ++(1,1);
    \filldraw[blue!40, rotate=30] (5,0) rectangle (6,1);
    %\filldraw[blue] (30:5) circle (0.1);
    \filldraw[blue,rotate=30] (6,1) circle (0.1);
    \draw[blue,dashed] (0,0) -- (6,1);
    \draw[blue,dashed, rotate=30] (0,0) -- (6,1);
    \draw[blue,dashed] (30:4) arc (30:40:4) node[xshift=0.5cm] {$\psi$};
  \end{tikzpicture}
\end{center}
\end{frame}

\begin{frame}
  \begin{gather*}
    \begin{bmatrix}
      x \\ y 
    \end{bmatrix} = r\begin{bmatrix}
      \cos\psi \\ \sin\psi
    \end{bmatrix} \\
    \text{ rodando no sentido anti-horário } \\
    R_{\theta}\begin{bmatrix}
      x \\ y
    \end{bmatrix} = r\begin{bmatrix}
      \cos(\theta + \psi) \\ \sin(\theta + \psi) 
    \end{bmatrix}= r\begin{bmatrix}
      \cos\theta \cos\psi - \sin\theta \sin\psi \\
      \cos\theta\sin\psi + \cos\psi\sin\theta
    \end{bmatrix} \implies \\
    R_{\theta}=
    \begin{bmatrix}
      x \\ y
    \end{bmatrix}=\begin{bmatrix}
      \cos\theta & -\sin\theta \\
      \sin\theta & \cos\theta 
    \end{bmatrix} \begin{bmatrix}
      r\cos\psi \\ r\sin\psi
    \end{bmatrix} =\begin{bmatrix}
      \cos\theta & -\sin\theta \\
      \sin\theta & \cos\theta 
    \end{bmatrix}\begin{bmatrix}
      x \\ y
    \end{bmatrix} 
  \end{gather*}

\end{frame}

\begin{frame} {Simetria em relação a uma reta}
  \begin{center}
    \begin{tikzpicture}
      \draw[gray] (-0.2,-0.2) grid (7.2,7.2);
      \draw[blue,-latex] (0,0)--(7.5,0);
      \draw[blue,-latex] (0,0)--(0,7.5);
      \draw[gray,dashed] (0,0) -- (30:7.5);
      %\filldraw[blue!40] (5,0) rectangle (6,1);
      %\filldraw[blue] (6,1) circle (0.1);
      \draw[gray,thick] (3,0) arc (0:30:3) node[anchor=north,yshift=-0.5cm] {$\theta$};
      \draw[blue,dashed,rotate=30] (0,0)--(4,1.5);
      \draw[red,dashed,rotate=30] (0,0) -- (4,-1.5);
      \filldraw[blue,rotate=30] (4,1.5) circle (0.1) node[anchor=east] {$\begin{bmatrix}
        x \\ y
      \end{bmatrix} $};
      \filldraw[red,rotate=30] (4,-1.5) circle (0.1) node[anchor=west] {$\begin{bmatrix}
        x^\prime \\ y^\prime
      \end{bmatrix} $};

    \end{tikzpicture}
  \end{center}
\end{frame}

\begin{frame}
  Se $\theta =0$ buscamos simplesmente a simetria em relação ao eixo $x$.
  $$ S_0\begin{bmatrix}
    x \\ y
  \end{bmatrix} = \begin{bmatrix}
    x \\ -y
  \end{bmatrix} = \begin{bmatrix}
    1 & 0 \\ 0 & -1
  \end{bmatrix}\begin{bmatrix}
    x \\ y
  \end{bmatrix}$$
  No caso geral: rodamos no sentido horário (ou outro, tanto faz), até a reta
  ficar paralela ao eixo $x$ aplicamos a simetria e rodamos de volta no outro sentido.
  Isto é, vamos usar que:
  $$ S_{\theta} = R_\theta S_0 R_{-\theta} $$
  então
  \begin{gather*}
    S_\theta \begin{bmatrix}
      x \\ y
    \end{bmatrix} = \begin{bmatrix}
      \cos\theta & \-sin\theta \\
      \sin\theta & \cos\theta
    \end{bmatrix}\begin{bmatrix}
      1 & 0 \\ 0 & -1 
    \end{bmatrix}\begin{bmatrix}
      \cos\theta & \sin\theta \\ 
      -\sin\theta & \cos\theta
    \end{bmatrix}\begin{bmatrix}
      x \\ y
    \end{bmatrix}
  \end{gather*}

\end{frame}

\begin{frame}
  \begin{gather*}
    S_\theta \begin{bmatrix}
      x \\ y
    \end{bmatrix} =\begin{bmatrix}
      \cos^2\theta -\sin^2\theta & 2\cos\theta \sin\theta \\
      2\sin\theta\cos\theta & -(\cos^2\theta -\sin^2\theta)
    \end{bmatrix}\begin{bmatrix}
      x \\ y
    \end{bmatrix} \\
    \text{ ou } \\
    S_\theta\begin{bmatrix}
      x \\ y
    \end{bmatrix}=\begin{bmatrix}
      \cos(2\theta) & \sin(2\theta)\\
      \sin(2\theta) & -\cos(2\theta)
    \end{bmatrix}\begin{bmatrix}
      x \\ y
    \end{bmatrix}
  \end{gather*}
\end{frame}

\begin{frame}{Projeção ortogonal}

  \begin{center}
    \begin{tikzpicture}
      \draw[gray] (-0.2,-0.2) grid (7.2,7.2);
      \draw[blue,-latex] (0,0)--(7.5,0);
      \draw[blue,-latex] (0,0)--(0,7.5);
      \draw[gray,dashed] (0,0) -- (30:7.5);
      %\filldraw[blue!40] (5,0) rectangle (6,1);
      %\filldraw[blue] (6,1) circle (0.1);
      \draw[gray,thick] (3,0) arc (0:30:3) node[anchor=north,yshift=-0.5cm] {$\theta$};
      \draw[blue,dashed,rotate=30] (0,0)--(4,1.5);
      %\draw[red,dashed,rotate=30] (0,0) -- (4,-1.5);
      \filldraw[blue,rotate=30] (4,1.5) circle (0.1) node[anchor=east] {$\begin{bmatrix}
        x \\ y
      \end{bmatrix} $};
      \filldraw[red,rotate=30] (4,0) circle (0.1) node[anchor=west] {$\begin{bmatrix}
        x^\prime \\ y^\prime
      \end{bmatrix} $};
      \draw[thin,-latex,rotate=30] (4,1.5) -- (4,0);

    \end{tikzpicture}
  \end{center}
  
\end{frame}

\begin{frame}
  \begin{gather*}
    P_\theta\begin{bmatrix}
      x \\ y
    \end{bmatrix}= a\begin{bmatrix}
      \cos\theta \\ \sin\theta
    \end{bmatrix} \text{ queremos achar }a \\
    \left(\begin{bmatrix}
      x \\ y
    \end{bmatrix} - a\begin{bmatrix}
      \cos\theta \\ \sin\theta
    \end{bmatrix}\right) \cdot \begin{bmatrix}
      \cos\theta \\ \sin\theta
    \end{bmatrix} = 0 \\
    a = x\cos\theta + y\sin\theta \\
    P_\theta\begin{bmatrix}
      x \\ y 
    \end{bmatrix} = \begin{bmatrix}
      x\cos^2\theta + y\sin\theta\cos\theta \\
      x\cos\theta\sin\theta +y\sin^2\theta
    \end{bmatrix} = \begin{bmatrix}
      \cos^2\theta & \sin\theta\cos\theta \\
      \sin\theta\cos\theta & \sin^2\theta
    \end{bmatrix}\begin{bmatrix}
      x \\ y
    \end{bmatrix}
  \end{gather*}
\end{frame}

\begin{frame}
  \begin{enumerate}
    \item A toda matriz está associada uma aplicação linear:
    \item A toda aplicação linear está associada uma matriz
  \end{enumerate}
  \begin{enumerate}
    \item $ A \mapsto T:\mathbb{R}^2 \to \mathbb{R}^2 \text{ como } T(\mathbf{x})=A\mathbf{x}$
    \item $ T:\mathbb{R}^2 \to \mathbb{R}^2 \mapsto A = [a_{ij}] \text{ como } a_{ij} = [T(\mathbf{e}_j)]_i$
  \end{enumerate}
\end{frame}


\begin{frame}{Rotações em $\mathbb{R}^3$}

  \begin{center}
    \begin{tikzpicture}
      \draw[-latex] (0,0,0) -- (0 , 0, 5) node[anchor=south] {$X$};
      \draw[dashed] (0,0,-5) -- (0,0,0);
      \draw[-latex] (0,0,0) -- (0 , 5, 0) node[anchor=east] {$Z$};
      \draw[dashed] (0,-2,0) -- (0,0,0);
      \draw[-latex] (0,0,0) -- (5 , 0, 0) node[anchor=north] {$Y$};
      \draw[dashed] (-5,0,0)-- (0,0,0);
      \draw[-latex,blue] (0,0,0)-- (3, 0, 4);
      \draw[-latex,blue] (0,0,0)-- (4, 0, -3);
      %agora o ângulo:
      \draw[red!70,very thick,-latex] (0,0,2) .. controls  (0.5,0,2.5) and (1,0,2.5) .. (1.2,0,1.6);
      \node[anchor=west,xshift=-1mm] at (0,-0.6,0) {$\theta$};
    \end{tikzpicture}
  \end{center}
  
\end{frame}

\begin{frame}
  \begin{gather*}
    R_\theta^3 = \begin{bmatrix}
      \cos\theta & -\sin\theta & 0 \\
      \sin\theta & \cos\theta & 0 \\
      0 & 0 & 1
    \end{bmatrix}
  \end{gather*}
\end{frame}

\begin{frame}
  \begin{center}
    \begin{tikzpicture}
      %eixos xyz
      \draw[thick,-latex] (0,0,0)--(0,0,2);
      \draw[thick,-latex] (0,0,0)--(2,0,0);
      \draw[thick,-latex] (0,0,0)--(0,2,0);
      \draw[thick,red,-latex](0,0,0)--(1.6,1.2,0);
      \draw[thick,red,-latex](0,0,0) --(-1.2,1.6,0); 
      \draw[red,dashed] (0,0,2)--(0,0,3);
      \draw[blue!60,-latex] (1,0,0) .. controls (1,0.1,0) and (1,0.4,0) .. (0.8,0.6,0);
      \node at (1.2,0.4,0) {$\varphi$};
      \draw[thick,-latex] (5,0,0)--(5,0,2);
      \draw[thick,-latex] (5,0,0)--(7,0,0);
      \draw[thick,-latex] (5,0,0)--(5,2,0);
      \draw[thick,red,-latex] (5,0,0) -- (5,1.2,1.6);
      \draw[thick,red,-latex] (5,0,0) -- (5,-1.6,1.2);
      \draw[red,dashed] (7,0,0) -- (8,0,0);
      \draw[thick,blue!60,-latex] (5,1,0) .. controls (5,1.2,0.4) and (5,1.2,0.6) .. (5,0.6,0.8);
      \node at (5,1.5,0.7) {$\psi$};
    \end{tikzpicture}
  \end{center}
\end{frame}

\begin{frame}
  \begin{gather*}
    R_\varphi^1 = \begin{bmatrix}
      1 & 0 & 0 \\
      0 & \cos\varphi & -\sin\varphi \\
      0 & \sin\varphi & \cos\varphi 
    \end{bmatrix}\text{ e }R_\psi^2 =\begin{bmatrix}
      \cos\psi & 0 & -\sin\psi \\
      0 & 1 & 0 \\
      \sin{\psi} & 0 & \cos\psi
    \end{bmatrix}
  \end{gather*}
\end{frame}

\begin{frame}{Ângulos de Euler}

  \begin{center}
    \begin{tikzpicture}
      \draw[-latex] (0,0,0) -- (0 , 0, 5) node[anchor=south] {$X$};
      \draw[dashed] (0,0,-5) -- (0,0,0);
      \draw[-latex] (0,0,0) -- (0 , 5, 0) node[anchor=east] {$Z$};
      \draw[dashed,gray!50] (0,-2,0) -- (0,0,0);
      \draw[-latex] (0,0,0) -- (5 , 0, 0) node[anchor=north] {$Y$};
      \draw[dashed] (-5,0,0)-- (0,0,0);
      % o outro referencial
      \draw[-latex,thick,red] (0,0,0)--(3.33, 1.67,3.33) node[anchor=west] {$X^\prime$};
      \draw[-latex,thick,red] (0,0,0) -- (-3.7,1.85,2.78) node[anchor=east] {$Z^\prime$};
      \draw[-latex,thick,red] (0,0,0) -- (0.24,4.36,-2.4) node[anchor=south] {$Y^\prime$};
      %linha nodal
      \draw[thick,dashed, green!60!blue] (0,0,0) -- (3,0,4) node {$N$};
      %% angulos
      \draw[very thick,red!60,-latex] (0,0,2) .. controls (0.4,0,2.0) and (0.8,0,2.5) .. (1.2,0,1.6); %theta
      \draw[very thick,blue!80,-latex] (1.8,0,2.4).. controls (2.0,0.2,2.5) and (2.2,0.8,2.5) ..(2,1,2);%varphi
      \draw[very thick, green!80,-latex] (0,2,0) .. controls (-0.5, 2 ,0.5) and (-1,2,0.6) .. (-1.48,0.7,1.1); %psi
      \node at (1.2,0,3) {$\theta$};
      \node at (2,0, 1.2) {$\varphi$};
      \node at (-1.3,1.8,0 ) {$\psi$};
    
    \end{tikzpicture}
  \end{center}
\end{frame}

\begin{frame}
  Podemos combinar estas rotações para escrever uma rotação genérica como:

  \begin{gather*}
    R =R_\varphi^3R_\psi^1 R_\theta^3= \\
   = \begin{bmatrix}
     \cos\varphi & -\sin\varphi & 0 \\
     \sin\varphi & \cos\varphi & 0 \\
     0 & 0 & 1 
    \end{bmatrix}\begin{bmatrix}
      1 & 0 & 0 \\
      0 & \cos\psi & -\sin\psi \\
      0 & \sin\psi & \cos\psi
    \end{bmatrix}\begin{bmatrix}
      \cos\theta & -\sin\theta & 0 \\
      \sin\theta & \cos\theta & 0 \\
      0 & 0 & 1
    \end{bmatrix}
  \end{gather*}
\end{frame}


\end{document}
