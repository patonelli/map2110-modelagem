\documentclass{beamer}
%
% Choose how your presentation looks.
%
% For more themes, color themes and font themes, see:
% http://deic.uab.es/~iblanes/beamer_gallery/index_by_theme.html
%
\mode<presentation>
{
  \usetheme{default}      % or try Darmstadt, Madrid, Warsaw, ...
  \usecolortheme{default} % or try albatross, beaver, crane, ...
  \usefonttheme{default}  % or try serif, structurebold, ...
  \setbeamertemplate{navigation symbols}{}
  \setbeamertemplate{caption}[numbered]
} 

\usepackage[brazil]{babel}
\usepackage[utf8]{inputenc}
\usepackage[T1]{fontenc}

\usepackage{tikz}
\usepackage{siunitx}

\title[Retas e planos]{Retas e planos em $\mathbb{R}^3$}
\author{MAP 2110 - Diurno}
\institute{IME USP}
\date{14 de julho de 2020}

\begin{document}



\begin{frame}
  \titlepage
\end{frame}

\begin{frame}{Equação de uma reta}

  \begin{itemize}
    \item $P$ é um ponto de $\mathbb{R}^3$
    \item $\mathbf{v} um vetor$
    \item O conjunto $\{ \mathbf{x} : x = P+t\mathbf{v}, t\in \mathbb{R}\}$ é uma reta em $\mathbb{R}^3$
    \item $r:P+t\mathbf{v}$ é a equação vetorial de uma reta.
    \item $\begin{bmatrix}
      x \\ y \\ z
    \end{bmatrix} = \begin{bmatrix}
      x_0 \\ y_0 \\ z_0
    \end{bmatrix}+ \begin{bmatrix}
      tv_1 \\ tv_2 \\ tv_3
    \end{bmatrix}$ é a equação paramétrica
  \end{itemize}
  
\end{frame}

\begin{frame}

  Para determinar a equação vetorial de uma reta, usamos um ponto $P$ e um vetor $\mathbf{v}$. Será que
  sempre que tomarmos $Q\neq P$ e $\mathbf{u} \neq \mathbf{v}$ as retas $P+t\mathbf{v}$ e $Q+s\mathbf{u}$ são diferentes?

\end{frame}

\begin{frame} {Problemas básicos}

  \begin{itemize}
    \item achar a intersecção de duas retas.
    \item projetar ortogonalmente um ponto sobre uma reta
  \end{itemize}
  
\end{frame}

\begin{frame}{Exemplo}
  Achar a intersecção das retas:
  \begin{gather*}
    \begin{array}{ccc}
      x &= &1-3t \\
      y & = & 2+5t\\
      z & = & 1+t
    \end{array}\text{ e }\begin{array}{ccc}
      x &=& -1+s \\
      y &=& 3-4s \\
      z &=& 1-s
    \end{array}
  \end{gather*}
\end{frame}

\begin{frame} {Exemplo 2}
  Achar a projeção ortogonal do ponto $Q = \begin{bmatrix}
    1 \\ 2 \\ 0
  \end{bmatrix}$ na reta $r: \begin{bmatrix}
    0 \\ 0 \\ 2
  \end{bmatrix}+ t\begin{bmatrix}
    1 \\ 1 \\ 1
  \end{bmatrix}$
\end{frame}

\begin{frame}{Equações vetoriais e paramétricas de um plano em $\mathbb{R}^3$}
  \begin{itemize}
    \item $P = \begin{bmatrix}
      p_1 \\ p_2 \\ p_2
    \end{bmatrix}$ é um ponto.
    \item $\mathbf{v}=\begin{bmatrix}
      v_1 \\ v_2 \\ v_3
    \end{bmatrix}$ e $\mathbf{u} =\begin{bmatrix}
      u_1 \\ u_2 \\ u_3
    \end{bmatrix}$ são vetores LI 
    \item $\Pi:P+t\mathbf{v} + s\mathbf{u}$ equação do plano.
    \item $\begin{array}{ccc}
    x&=&p_1 + tv_1 + su_1 \\
    y&=& p_2 + tv_2 + su_2 \\
    z&=& p_3 + tv_3 + su_3
  \end{array}$ é a equação paramétrica
  \end{itemize}
\end{frame}

\begin{frame}{Direção normal a um plano}

  O vetor $\mathbf{n}=\mathbf{v}\times \mathbf{u}$ determina um vetor perpendicular ao plano $\Pi$. Dizem os que $\mathbf{n}$ é normal ao plano
  $\Pi$ que pode ser caracterizado como os pontos $X$ tais que $(X-P)\cdot \mathbf{n}=0$, chamada de equação vetorial do plano.

  Se temos $\mathbf{n} = \begin{bmatrix}
    a \\ b \\ c
  \end{bmatrix}$ e $P = \begin{bmatrix}
    p_1 \\ p_2 \\ p_2
  \end{bmatrix}$  um ponto do plano. Então as coordenadas de qualquer outro ponto do plano satisfaz:
  \begin{gather*}
    \left(\begin{bmatrix}
      x \\ y \\ z
    \end{bmatrix} - \begin{bmatrix}
      p_1 \\ p_2 \\ p_3
    \end{bmatrix}\right)\cdot \begin{bmatrix}
      a \\ b \\ c 
    \end{bmatrix}= 0 \implies \\
    ax+by+cz = F \\
    F = P\cdot \mathbf{n}
  \end{gather*}
  A última equação é a equação geral de um plano no espaço.
\end{frame}

\begin{frame}{problemas básicos}
Intersecção de dois planos.
\begin{gather*}
  \left\{ \begin{array}{lccc}
    \Pi_1: & x+y+z & = & 1 \\
    \Pi_2: & 2x-y-2z& = & 3
  \end{array}\right. \\
  \begin{bmatrix}
    1 & 1 & 1 \\
    2 & -1 & -2
  \end{bmatrix}\begin{bmatrix}
    x \\ y \\ z
  \end{bmatrix} = \begin{bmatrix}
    1 \\ 3
  \end{bmatrix} \\
  \begin{array}{|ccc|c|}\hline
    1 & 1 & 1 & 1 \\
    2 & -1 & -2 & 3 \\ \hline
  \end{array} \to \begin{array}{|ccc|c|}\hline
    1 & 1 & 1 & 1 \\
    0 & -3 & -4 & 1 \\ \hline
  \end{array} \to \\
  \begin{array}{|ccc|c|}\hline
    1 & 1 & 1 & 1 \\
    0 & 1 & 4/3& -1/3\\ \hline
  \end{array} \to \begin{array}{|ccc|c|}\hline
    1 & 0 & -1/3 & 4/3 \\
    0 & 1 & 4/3& -1/3\\ \hline
  \end{array} 
\end{gather*}
  
\end{frame}

\begin{frame}
  Projetar um ponto sobre um plano

\textbf{Exemplo:} Projetar o ponto $P = \begin{bmatrix}
  2 \\ 2 \\ 3
\end{bmatrix}$ ortogonalmente no plano $x+y+z=1$

\end{frame}

\end{document}
