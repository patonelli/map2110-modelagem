\documentclass{beamer}
%
% Choose how your presentation looks.
%
% For more themes, color themes and font themes, see:
% http://deic.uab.es/~iblanes/beamer_gallery/index_by_theme.html
%
\mode<presentation>
{
  \usetheme{default}      % or try Darmstadt, Madrid, Warsaw, ...
  \usecolortheme{default} % or try albatross, beaver, crane, ...
  \usefonttheme{default}  % or try serif, structurebold, ...
  \setbeamertemplate{navigation symbols}{}
  \setbeamertemplate{caption}[numbered]
} 

\usepackage[english]{babel}
\usepackage[utf8]{inputenc}
\usepackage[T1]{fontenc}

\title[Retas e Planos]{Retas e Planos em $V_3$}
\author{MAP 2110 - Diurno}
\institute{IME USP}
\date{31 de março}

\begin{document}

\begin{frame}
  \titlepage
\end{frame}

% Uncomment these lines for an automatically generated outline.
%\begin{frame}{Outline}
%  \tableofcontents
%\end{frame}

\section {Exercícios}

\begin{frame}{Exercícios}
 
 Faremos uma seleção de exercícios do Apostol

\end{frame}

\section{Alguns Exercícios}

\subsection{Exercícios do Apostol}

\begin{frame}{Exercicio 1:}
Uma reta $L$ em $V_2$ contém os pontos $P=(-3,1)$ e $Q=(1,1),$ quais dos seguintes pontos também estão em $L$
\begin{itemize}
   \item[A]$(0,0)$
   \item[B]$(0,1)$
   \item[C]$(1,2)$
   \item[D]$(2,1)$
   \item[E]$(-2,1)$
\end{itemize}
\end{frame}

\subsection{Exercicio 2}

\begin{frame}{Exercicio 2}

Verifique em cada um dos casos se os três pontos estão numa mesma reta:
\begin{itemize}
    \item $P=(2,1,1)$ $Q=(4,1,-1)$ e $R=(3,-1,1)$
    \item $P=(2,2,3)$ $Q=(-2,3,1)$ e $R=(-6,4,1)$
    \item $P=(2,1,1)$ $Q=(-2,3,1)$ e $R=(5,-1,1)$
\end{itemize}

\end{frame}

\begin{frame}{Exercício 3}
   Uma reta $L_1$ passa pelo ponto $P=(1,1,1)$ e é paralela ao vetor $A=(1,2,3).$ Uma outra $L_2$ passa pelo ponto $Q=(2,1,0)$ e é paralela ao vetor $B=(3,8,13).$  Mostre que as duas retas se interceptam e determine o ponto de intersecção.
\end{frame}

\begin{frame}{Exercício 4}
Seja $X(t) = P + tA$ um ponto genérico da reta $L(P,A)$ onde $P=(1,2,3)$ e $A=(1,-2,2).$ Tomemos o ponto $Q=(3,3,1)$
\begin{itemize}
    \item Calcule $\| X(t) - Q \|^2$
    \item Mostre que existe um único ponto $X(t_0)$ que minimiza $\| X(t) - Q \|^2$
    \item Mostre que $Q-X(t_0)$ é ortogonal a $A$
\end{itemize}
    \end{frame}

\end{document}
