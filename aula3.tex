\documentclass{beamer}
%
% Choose how your presentation looks.
%
% For more themes, color themes and font themes, see:
% http://deic.uab.es/~iblanes/beamer_gallery/index_by_theme.html
%
\mode<presentation>
{
  \usetheme{default}      % or try Darmstadt, Madrid, Warsaw, ...
  \usecolortheme{default} % or try albatross, beaver, crane, ...
  \usefonttheme{default}  % or try serif, structurebold, ...
  \setbeamertemplate{navigation symbols}{}
  \setbeamertemplate{caption}[numbered]
} 

\usepackage[english]{babel}
\usepackage[utf8]{inputenc}
\usepackage[T1]{fontenc}

\title[Retas e Planos]{Produto Vetorial e Cônicas}
\author{MAP 2110 - Diurno}
\institute{IME USP}
\date{7 de abril}

\begin{document}

\begin{frame}
  \titlepage
\end{frame}

% Uncomment these lines for an automatically generated outline.
%\begin{frame}{Outline}
%  \tableofcontents
%\end{frame}

\section {Produto Vetorial}

\begin{frame}{Definição do Produto Vetorial }
 
 \begin{gather*}
   \vec{a}=(a_1,a_2,a_3)\\
   \vec{b}=(b_1,b_2,b_3)\\
   \vec{a}\times \vec{b}=(a_2b_3-a_3b_2, a_3b_1-a_1b_3, a_1b_2-a_2b_1)
 \end{gather*}

\end{frame}

\begin{frame}{Exemplos}
Uma reta $L$ em $V_2$ contém os pontos $P=(-3,1)$ e $Q=(1,1),$ quais dos seguintes pontos também estão em $L$
\begin{itemize}
   \item[A]$(0, 1, 2) \times (1,0,-1)$
   \item[B]$ (\mathbf{i}\times\mathbf{k})\times\mathbf{j}$
   \item[C]$ (\mathbf{i}\times\mathbf{i})\times\mathbf{j}$
   \item[D]$ (\mathbf{i} + 2\mathbf{j}) \times (2\mathbf{i} - \mathbf{k}$)
\end{itemize}
\end{frame}
\begin{frame}
  

\end{frame}

\begin{frame}{Propriedades}
  \begin{itemize}
    \item $(\vec{a}\times \vec{b}) = - (\vec{b}\times \vec{a})$
    \item $\vec{a}\times(\vec{b} + \vec{c})=\vec{a}\times \vec{b} + \vec{a}\times\vec{c}$
    \item $\alpha(\vec{a}\times \vec{b})=\alpha\vec{a}\times \vec{b}$
    \item $\|\vec{a}\times\vec{b}\|^2 =\|\vec{a}\|^2 \|\vec{b}\|^2 -(\vec{a}\cdot\vec{b})^2$  \end{itemize}

\end{frame}
\begin{frame}
  
\end{frame}


\begin{frame}{Exercício}
  Sejam dados os vetores $\vec{a}=2\mathbf{i}-\mathbf{j}+2\mathbf{k}$ e 
  $\vec{c}=3\mathbf{i} +4\mathbf{j} -\mathbf{k}$, encontrar um vetor $\vec{b}$ tal que
  $\vec{a}\times \vec{b}=\vec{c}.$ Esta solução é única?
\end{frame}

\begin{frame}{solução}
Seja $\vec{b}=b_1\mathbf{i}+b_2\mathbf{j} + b_3\mathbf{k}$
Então podemos escrever usando a propriedade distributiva :
\begin{gather*}
  \vec{a}\times \vec{b}=\vec{a}\times b_1\mathbf{i} + \vec{a}\times b_2\mathbf{j} + 
  \vec{a}\times b_3\mathbf{k}\\
  \text{ usamos que } \mathbf{i}\times\mathbf{j} = \mathbf{k}\text{ } \mathbf{j}\times\mathbf{k} = \mathbf{i}\text{ }
  \mathbf{k}\times\mathbf{i} = \mathbf{j}\\
  \vec{a}\times \vec{b} = (-b_3-2b_2)\mathbf{i} + (2b_1-2b_3)\mathbf{j} +(b_1 +2b_2)\mathbf{k} \\
\end{gather*}
\end{frame}

\begin{frame}
  Comparando os vetores temos o sistema
  \begin{gather*}
    -2b_2 - b_3 = 3 \\
    2b_1 - 2b_3 = 4 \\
    b_1 + 2b_2 = -1
  \end{gather*}
  O sistema é indeterminado e podemos escrever as soluções
  como $\vec{b}= (2\mathbf{i} - \frac{3}{2}\mathbf{j}) + b_3(\mathbf{i}-\frac{1}{2}\mathbf{j}
  +\mathbf{k})$

  Para que $\vec{a}\cdot \vec{b}=1$ a solução é única ($b_3=\frac{-11}{9}$)

  
\end{frame}

\section{Cônicas}

\begin{frame}{Definição}
  O Apostol apresenta três possíveis definições de cônicas, e todos são equivalentes.
  Mas vamos usar a definição que faz mais uso do conceito de vetor.Nosso situação agora 
  num plano. Então podemos fazer todas as contas em $V_2$.

  \textbf{Definição:} Se $L$ é uma reta em $V_2$, $F$ é um ponto fora de $L$ e $e>0$ um 
  número real positivo, então o conjunto:
  $$ C=\{X : \|X-F\| = e d(X,L)    \}$$ é uma cônica, e diremos que $C$ é uma elípse se 
  $e<1$, uma parábola se $e=1$ e uma hipérbole se $e>1$

\end{frame}

\begin{frame}{Como expressar $d(X,L)$}
  
  
 

\end{frame}

\begin{frame}{Exercício 4}

\end{frame}

    \begin{frame}{solução}
     
    \end{frame}
   
    \begin{frame}{Equação do Plano}
     
      
    \end{frame}
    \begin{frame}{Solução}
      
    \end{frame}
\begin{frame}
  
\end{frame}
\begin{frame}{solução} 
  
\end{frame}

\end{document}
