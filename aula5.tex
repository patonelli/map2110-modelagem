\documentclass{beamer}
%
% Choose how your presentation looks.
%
% For more themes, color themes and font themes, see:
% http://deic.uab.es/~iblanes/beamer_gallery/index_by_theme.html
%
\mode<presentation>
{
  \usetheme{default}      % or try Darmstadt, Madrid, Warsaw, ...
  \usecolortheme{default} % or try albatross, beaver, crane, ...
  \usefonttheme{default}  % or try serif, structurebold, ...
  \setbeamertemplate{navigation symbols}{}
  \setbeamertemplate{caption}[numbered]
} 

\usepackage[english]{babel}
\usepackage[utf8]{inputenc}
\usepackage[T1]{fontenc}

\title[Sistemas Lineares]{Sistemas Lineares e Método da eliminação de Gauss}
\author{MAP 2110 - Diurno}
\institute{IME USP}
\date{23 de abril}

\begin{document}

\begin{frame}
  \titlepage
\end{frame}

% Uncomment these lines for an automatically generated outline.
%\begin{frame}{Outline}
%  \tableofcontents
%\end{frame}


\section{Sistemas Lineares}

\begin{frame}{Sistemas lineares}

  Um sistema linear, é um conjunto de equações nas variáveis $x_1,\dots x_n$ que devem ser resolvidas
   simultaneamente na forma:
   \begin{block}{Sistema Linear}
     \begin{gather*}
      a_{11}x_1 + a_{12}x_2 + \cdots + a_{1n}x_n = b_1 \\
      a_{21}x_1 + a_{22}x_2 + \cdots + a_{2n}x_n = b_2 \\
      \vdots \\
      a_{m1}x_1 + a_{m2}x_2 + \cdots + a_{mn}x_n = b_m \\
    \end{gather*}
  \end{block}

   Os números $a_{ij}$ são chamados coeficientes do sistema linear.
\end{frame}

\begin{frame}{Exemplo}
  \begin{gather*}
    2x_1 + 3x_2 + x_3 = 3 \\
    x_1 -2x_2 -2x_3 = -1 \\
    0x_1 + x_2 + x_3 = 1
  \end{gather*}
  Tem uma única solução
  $x_1=1, x_2=0, x_3=1$ 
Problema: como resolver um sistema linear qualquer?
\end{frame}

\begin{frame}{quantidade de incógnitas e equações}
  \begin{tabular}{|c|cccc|c|} \hline 
    & $x_1$ & $x_2$ & $\cdots $ & $x_n$ & b \\ \hline
    $E_1$ & $a_{11}x_1$ & $+a_{12}x_2$ & $\cdots $ &$+a_{1n}x_n$ &$b_1$ \\
    $E_2$ & $a_{21}x_1$ & $+a_{22}x_2$& $\cdots $ &$+a_{2n}x_n $&$b_2$ \\
          &     $\vdots$       &    $\vdots$         &         & $\vdots$ &  \\
    $E_m$ & $a_{m1}x_1$ & $+a_{m2}x_2$& $\cdots $ &$+a_{mn}x_n$ &$b_m$ \\ \hline
  \end{tabular}

  De forma geral a quantidade de incógnitas pode ser diferente do número de equações ($m\neq n$).
  O que pode afetar a quantidade de soluções e a existência delas. Nós vamos primeiro analisar o caso em que
  o número de equações e incógnitas são o mesmo, digamos $n$.

\end{frame}

\begin{frame}{Classe de sistemas simples de resolver}
Embora a resolução dos sistemas lineares envolvam montes de contas, em geral. Existem uma classe de sistemas que
mais simples de resolver. Por exemplo:
$$
\begin{array}{ccccc}
  2x_1 + & 3x_2 + & x_3 & = & 3 \\
          & 2x_2 + & x_3 & = & 1 \\
          &         & x_3 & = & 1
\end{array}$$

\end{frame}


\begin{frame}{ Sistemas Triangulares Superiores}
  $$
\begin{array}{cccccc}
  a_{11}x_1 + & a_{12}x_2 + & \cdots & a_{1n}x_n & = & b_1 \\
          & a_{22}x_2 + & \cdots&a_{2n}x_n & = & b_2 \\
          &              & \ddots & & & \\
          &       &  & a_{nn}x_n & = & b_n
\end{array}$$
Estes sistemas conseguimos resolver se todos os $a_{nn}$ forem diferentes de zero. basta usar a fórmula
$$ x_k = \frac{1}{a_{kk}}(b_k - \sum_{i=k+1}^n a_{ki}x_i)$$ começando com $k=n$ e voltanto até $1$. 
Um procedimento chamado de Backward Substituition
\end{frame}

\begin{frame}
Num sistema linear 

\begin{tabular}{|c|cccc|c|} \hline 
  & $x_1$ & $x_2$ & $\cdots $ & $x_n$ & b \\ \hline
  $E_1$ & $a_{11}x_1$ & $+a_{12}x_2$ & $\cdots $ &$+a_{1n}x_n$ &$b_1$ \\
  $E_2$ & $a_{21}x_1$ & $+a_{22}x_2$& $\cdots $ &$+a_{2n}x_n $&$b_2$ \\
        &     $\vdots$       &    $\vdots$         &         & $\vdots$ &  \\
  $E_m$ & $a_{m1}x_1$ & $+a_{m2}x_2$& $\cdots $ &$+a_{mn}x_n$ &$b_m$ \\ \hline
\end{tabular}

Podemos fazer algumas operações nas equações $E_i$ de forma que não alteramos o conjunto solução do sistema.

\pause 

Destacaremos aqui três destas operações que chamaremos de \textbf{Operações Elementares} 
\begin{itemize}
  \item Troca das linhas da equação que denotaremos por $E(i,j)$ 
  \item Multiplicar uma equação por um escalar diferente de zero ($E(i;\alpha)$)
  \item Somar a uma equação o múltiplo de uma outra equação.

\end{itemize}
\end{frame}

\begin{frame}{Exemplo}
\begin{block}{Troca de linhas}
  \begin{gather*}
    \begin{array}{ccccc}
      2x_1 & + 3x_2& + x_3& =& 3 \\
      x_1 & -2x_2 &-2x_3&=& -1
    \end{array} (E(1,2)) \rightarrow \begin{array}{ccccc}
      x_1 & -2x_2 &-2x_3&=& -1 \\
      2x_1 & + 3x_2& + x_3& =& 3 
    \end{array} 
  \end{gather*}
\end{block}

\begin{block}{Multiplicar linha por escalar}
  \begin{gather*}
    \begin{array}{ccccc}
      x_1 & -2x_2 &-2x_3&=& -1 \\
      2x_1 & + 3x_2& + x_3& =& 3 
    \end{array} \rightarrow \begin{array}{ccccc}
      x_1 & -2x_2& -2x_3& =& -1 \\
      x_1 & +1.5x_2 &+0.5x_3&=& 1.5
    \end{array}  
  \end{gather*}
\end{block}

\begin{block}{Somar uma linha com múltiplo de outra}
  \begin{gather*}
    \begin{array}{ccccc}
      x_1 & -2x_2& -2x_3& =& -1 \\
      x_1 & +1.5x_2 &+0.5x_3&=& 1.5
    \end{array}  \rightarrow
    \begin{array}{ccccc}
      x_1 & -2x_2& -2x_3& =& -1 \\
          & 3.5x_2 &+2.5x_3&=& 2.5
    \end{array}  
  \end{gather*}
\end{block}
\end{frame}


\begin{frame} 
  Será que usando as operações elementares sempre conseguiremos
  deixar um sistema linear na forma triangular superior?
  Vamos começar com um exemplo:
  \begin{gather*}
    2x_1 + x_3 -x_4 = -6\\
    x_1 + x_2 + x_4 = 6 \\
    -x_1 +2x_2+3x_3+6x_4=38 \\
    5x_1 + 2x_2 + 10x_3 +x_4 =14
  \end{gather*}




\end{frame}

\begin{frame}
  \begin{gather*}
    2x_1 + x_3 -x_4 = -6\\
     x_2 -0.5x_3+ 1.5x_4 = 9 \\
     2x_2+3.5x_3+5.5x_4=35 \\
     2x_2 + 7.5x_3 +3.5x_4 =29
  \end{gather*}
\end{frame}


\begin{frame}{ }
  \begin{gather*}
    2x_1 + x_3 -x_4 = -6\\
     x_2 -0.5x_3+ 1.5x_4 = 9 \\
     4.5x_3+2.5x_4=17 \\
     8.5x_3 +0.5x_4 =11
  \end{gather*}
\end{frame}

\begin{frame}{ }
  \begin{gather*}
    2x_1 + x_3 -x_4 = -6\\
     x_2 -0.5x_3+ 1.5x_4 = 9 \\
     4.5x_3+2.5x_4=17 \\
      \frac{-38}{9}x_4 = \frac{-190}{9}
  \end{gather*}
\end{frame}

\begin{frame}{Matrizes do sistema linear}
  Note que as operações elementares alteram os coeficientes
  do sistema linear, assim como os elementos do lado direito da igualdade.
  Assim executar as operações elementares nas equações 
  do sistema linear é equivalente e realizar as operações
  elementares nas linhas da \textbf{matriz do sistema linear}
  $$
  \left[\begin{array}{cccc|c}
    a_{11} & a_{12} & \cdots & a_{1n} & b_1 \\
    a_{21} & a_{22} & \cdots & a_{2n} & b_2 \\
    \vdots &        &        & \vdots & \vdots \\
    a_{m1} & a_{m2} & \cdots & a_{mn} & b_m 
  \end{array}\right]
  $$
\end{frame}

\begin{frame}{Exemplo}
  Para o sistema anterior teríamos:
  $$
  \left[\begin{array}{cccc|c}
    2 & 0 & 1 & -1 & -6 \\
    1 & 1 & 0 & 1 & 6 \\
    -1 &   2     &    3    & 6 & 38 \\
    5 & 2 & 10 & 1 & 14
  \end{array}\right] \rightarrow \left[\begin{array}{cccc|c}
    2 & 0 & 1 & -1 & -6 \\
    0 & 1 & -0.5 & 1.5 & 9 \\
    0 &   2     &    3.5    & 5.5 & 35 \\
    0 & 2 & 7.5 & 3.5 & 29
\end{array}\right] \rightarrow
  $$
  $$
  \left[\begin{array}{cccc|c}
    2 & 0 & 1 & -1 & -6 \\
    0 & 1 & -0.5 & 1.5 & 9 \\
    0 &   0     &    4.5    & 2.5 & 17 \\
    0 & 0 & 8.5 & 0.5 & 11
  \end{array}\right] \rightarrow \left[\begin{array}{cccc|c}
    2 & 0 & 1 & -1 & -6 \\
    0 & 1 & -0.5 & 1.5 & 9 \\
    0 &   0     &    4.5    & 2.5 & 17 \\
    0 & 0 & 0 & -38/9 & -190/9
\end{array}\right]
  $$
  
\end{frame}
   
\end{document}
