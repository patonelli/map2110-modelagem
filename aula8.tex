\documentclass{beamer}
%
% Choose how your presentation looks.
%
% For more themes, color themes and font themes, see:
% http://deic.uab.es/~iblanes/beamer_gallery/index_by_theme.html
%
\mode<presentation>
{
  \usetheme{default}      % or try Darmstadt, Madrid, Warsaw, ...
  \usecolortheme{default} % or try albatross, beaver, crane, ...
  \usefonttheme{default}  % or try serif, structurebold, ...
  \setbeamertemplate{navigation symbols}{}
  \setbeamertemplate{caption}[numbered]
} 

\usepackage[english]{babel}
\usepackage[utf8]{inputenc}
\usepackage[T1]{fontenc}

\usepackage{tikz}

\title[Inversas]{Matrizes Inversas}
\author{MAP 2110 - Diurno}
\institute{IME USP}
\date{12 de maio}

\begin{document}

\begin{frame}
  \titlepage
\end{frame}

% Uncomment these lines for an automatically generated outline.
%\begin{frame}{Outline}
%  \tableofcontents
%\end{frame}


\section{Inversas de Matrizes}

\begin{frame}{Problema da Inversão}

  $I_n =[\delta_{ij}]$ é a matriz identidade $n\times n$. Ela tem a seguinte propriedade importante:
  para toda matriz $A$ de dimensão $m \times n$ vale:
  $$ I_m A = A \text{ e } AI_n =A $$ ou seja, ela é um elemento neutro na multiplicação de matrizes.
  O problema que podemos colocar é será que para esta matriz $A$ existe uma matriz $B$ de dimensão $n \times m$ tal que
  $$ A.B = I_m \text { e } B.A = I_n \text{ (?)}$$
  
 
\end{frame}

\begin{frame}{Exemplo}
  Vamos achar uma solução para
  \begin{gather*}
    \begin{bmatrix}
      1 & 0 & 1 \\
      1 & -1 & 1
    \end{bmatrix}B= \begin{bmatrix}
      1 & 0 \\ 0 & 1
    \end{bmatrix}
  \end{gather*}
  $$\begin{array}{|ccc|cc|} 
    1 & 0 & 1 & 1 & 0 \\
    1 & -1 & 1 & 0 & 1
  \end{array} \rightarrow 
  \begin{array}{|ccc|cc|} 
    1 & 0 & 1 & 1 & 0 \\
    0 & -1 & 0 & -1 & 1
  \end{array} \rightarrow \cdots
  $$
  $$\begin{array}{|ccc|cc|} 
    1 & 0 & 1 & 1 & 0 \\
    0 & 1 & 0 & 1 & -1
  \end{array} \implies B= \begin{bmatrix}
    1-t & -s \\ 1 & -1 \\ t & s
  \end{bmatrix} $$

  
\end{frame}

\begin{frame}{}
  Agora vamos verificar se 
  $$ 
  \begin{bmatrix}
    1-t & -s \\ 1 & -1 \\ t & s
  \end{bmatrix}\begin{bmatrix}
    1 & 0 & 1 \\
    1 & -1 & 1
  \end{bmatrix}=I_3$$
  $$ \begin{bmatrix}
    1-t-s & s & 1 -t -s \\
    0 & 1 & 0 \\
    t+s & -s & t+s
  \end{bmatrix} = I_3 \text{ não é possível}$$

\end{frame}

\begin{frame}{Definição da matriz inversa}

De forma geral não é possível definir uma inversa para matrizes $m\times n$ se $m\neq n$. Vamos definir então para
matrizes quadradas $n\times n$:

Uma matriz $A$ é invertível se existe uma matriz $B$ de mesma dimensão (claro) tal que:
$$ AB=BA = I_n $$
Uma matriz $B$ que satisfaz esta condição chamaremos de inversa de $A$ e denotaremos por $A^{-1}$
  
  \end{frame} 


\begin{frame}{Unicidade da inversa }
  É importante que a inversa seja única. Então:
  Se $B$ e $C$ são duas matrizes satisfazendo as condições da inversa de $A$ então temos
  $$ C= CI_n = C(AB)=(CA)B = I_nB=B $$
 
  
\end{frame}

\begin{frame}{ Exemplo }

  \begin{gather*}
   A= \begin{bmatrix}
      2 & 1 \\ 1 & 1 
    \end{bmatrix} \text{ então } A^{-1}=\begin{bmatrix}
      1 & -1 \\ -1 & 2
    \end{bmatrix} \\
    A= \begin{bmatrix}
      0 & 1 \\
      0 & -2 
    \end{bmatrix} \text{ então  não existe } A^{-1}
  \end{gather*}
 nem todas as matrizes têm inversas 
\end{frame}

\begin{frame}{Como tentar achar a inversa de uma matriz}
Achar a matriz inversa de 
$$ A = \begin{bmatrix}
  1 & 2 & 0 \\
  2 & 1 & 1 \\
  -1 & 0 & 0
\end{bmatrix}$$
Eliminação de Gauss (e Jordan)
$$ 
\begin{array}{|ccc|ccc|c}
  1&2&0&1&0&0& \\
  2&1&1&0&1&0& \textcolor{red}{L_2 -2L_1}\\
  -1&0&0&0&0&1&\textcolor{red}{L_3 + L_1}
\end{array}\rightarrow
$$
\end{frame}

\begin{frame}{Como antes}
  \begin{gather*} 
  \begin{array}{|ccc|ccc|c}
    1&2&0&1&0&0& \\
    0&-3&1&-2&1&0& \\
    0&2&0&1&0&1&\textcolor{red}{L_3 + \frac{2}{3}L_2}
  \end{array}\rightarrow\\
\begin{array}{|ccc|ccc|c}
    1&2&0&1&0&0& \\
    0&-3&1&-2&1&0& \textcolor{red}{-1/3L_2} \\
    0&0&2/3&-1/3&2/3&1& \textcolor{red}{3/2L_3}
  \end{array}\rightarrow\\
  \begin{array}{|ccc|ccc|c}
    1&2&0&1&0&0& \\
    0&1&-1/3&2/3&-1/3&0& \textcolor{blue}{L_2+1/3L_3} \\
    0&0&1&-1/2&1&3/2& 
  \end{array}\rightarrow
\end{gather*}
 \end{frame}

\begin{frame}{Agora a parte pra achar a escalonada reduzida}
\begin{gather*}
  \begin{array}{|ccc|ccc|c}
    1&2&0&1&0&0&\textcolor{blue}{L_1-2L_2} \\
    0&1& 0& 1/2&0& 1/2&  \\
    0&0&1&-1/2&1&3/2& 
  \end{array}\rightarrow \\
    \begin{array}{|ccc|ccc|c}
      1&0&0&0&0&-1&\\\
      0&1& 0& 1/2&0& 1/2&  \\
      0&0&1&-1/2&1&3/2& 
    \end{array}\rightarrow\\
    A^{-1} = \frac{1}{2}\begin{bmatrix}
      0 & 0 & -2 \\
      1 & 0 & 1 \\
      -1 & 2 & 3
    \end{bmatrix}
  \end{gather*}
\end{frame}

\begin{frame}{teste}
  $$
  \begin{array}{|ccc|ccc|} \hline
    & & &0 &0 & -2\\
    & &\frac{1}{2} \times &1 &0 & 1\\
    & & &-1 & 2& 3 \\ \hline
   1 &2 & 0&1 & 0& 0\\
   2 & 1& 1& 0& 1& 0\\
   -1 & 0& 0&0 & 0& 1\\ \hline
  \end{array}
  $$
\end{frame}
\begin{frame}{Falso ou Verdadeiro}

  \begin{block}{1}
 Uma matriz $A$ de dimensão $n\times n$ é invertível
 se e somente se a forma escalonada reduzida é a identidade.
  \end{block}

  \textcolor{blue}{ Verdadeiro, basta estudar o exemplo acima. Só não haverá inversa quando a forma escalonada tiver
  uma linha de zeros}
  
\end{frame}

\begin{frame}
  Lista das operações elementares que fizemos acima:
  \begin{itemize}
    \item $L_2 -2L_1$
    \item $L_3 + L_1$
    \item $L_3 + 3/2L_2$
    \item $-1/3L_2$
    \item $3/2L_3$
    \item $L_2+1/3L_3$
    \item $L_1 -L_2$
  \end{itemize}
\end{frame}

\begin{frame}{correspondem às matrizes elementares}
  \begin{gather*}
1-  \begin{pmatrix}
    1 & 0 & 0 \\ 
    -2 & 1 & 0 \\
    0 & 0 & 1
  \end{pmatrix}\text{ }2- \begin{pmatrix}
    1 & 0 & 0 \\ 
    0 & 1 & 0 \\
    1 & 0 & 1
  \end{pmatrix}\text{ }3- \begin{pmatrix}
    1 & 0 & 0 \\ 
    0 & 1 & 0 \\
    0 & 3/2 & 1
  \end{pmatrix} \\
  \text{ }4- \begin{pmatrix}
    1 & 0 & 0 \\ 
    0 & -1/3 & 0 \\
    0 & 0 & 1
  \end{pmatrix}\text{ }5- \begin{pmatrix}
    1 & 0 & 0 \\ 
    0 & 1 & 0 \\
    0 & 0 & 3/2
  \end{pmatrix}\text{ }6- \begin{pmatrix}
    1 & 0 & 0 \\ 
    0 & 1 & 1/3 \\
    0 & 0 & 1
  \end{pmatrix} \\
  \text{ }7- \begin{pmatrix}
    1 & -1 & 0 \\ 
    0 & 1 & 0 \\
    0 & 0 & 1
  \end{pmatrix}
  \end{gather*}
\end{frame}





\begin{frame}{Inversa de uma matriz $2\times 2$}
  Se $A=\begin{bmatrix}
    a & b \\ c & d
  \end{bmatrix}$ é uma matriz vamos chamar de 
  $\det(A) = ad -cb$. Se $\det(A)\neq 0$ 
  $$ A^{-1}=\frac{1}{\det(A)} \begin{bmatrix}
    d & -b \\
    -c & a
  \end{bmatrix}$$
\end{frame}

\begin{frame}{Verdadeiro ou Falso}
  \begin{block}{2}
  Se $A$ e $B$ são duas matrizes invertíveis de dimensão
  $n\times n$ então $(AB)^{-1} = A^{-1}B^{-1}$

  \textcolor{blue}{Falso. A fórmula correta é $(AB)^{-1}=B^{-1}A^{-1}$}
\end{block}
\end{frame}



\begin{frame}{ Matrizes Elementares são invertíveis}

$$ A= \begin{bmatrix}
  1 & 0 & 0 \\
  2 & 1 & 0 \\
  0 & 0 & 1
\end{bmatrix}$$
qual é a inversa de $A$ \pause 
\textcolor{blue}{
$$ A= \begin{bmatrix}
  1 & 0 & 0 \\
  -2 & 1 & 0 \\
  0 & 0 & 1
\end{bmatrix}$$}
\end{frame}


\begin{frame}{Verdadeiro ou Falso}
  \begin{block}{3}
 Se $A$ é uma matriz $n\times n$ invertível então a equação 
 $$ A\mathbf{x} = \mathbf{b} $$
 tem uma única solução para qualquer $\mathbf{b}$ de dimensão $n\times 1$, ou seja, um vetor né?
\end{block}

\textcolor{blue}{Verdadeiro: $x = A^{-1}b$ um tipo de lei do cancelamento.}
\end{frame}


\begin{frame}{Verdadeiro ou Falso}
  \begin{block}{4}
    Se $A$ é uma matriz $m\times n$, então $A.A^T$ é uma matriz quadrada 
  
    \textcolor{blue}{Verdadeiro. Se você não entendeu por quê, leia novamente a primeira seção do capítulo 2 do 
    Nicholson}
\end{block}
\end{frame}   

\begin{frame}{Verdadeiro ou Falso}
  \begin{block}{5}
  Se $A$ é uma matriz $2\times 2$ e $A^2=-I_2$ então $A$ tem algum elemento complexo.
\end{block}

\textcolor{blue}{Falso. A matriz $A=\begin{pmatrix}
  0 & 1 \\ -1 & 0 
\end{pmatrix}$ satisfaz essa propriedade embora só tenha elementos reais.}
\end{frame}



\end{document}
