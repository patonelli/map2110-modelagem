\documentclass{beamer}
%
% Choose how your presentation looks.
%
% For more themes, color themes and font themes, see:
% http://deic.uab.es/~iblanes/beamer_gallery/index_by_theme.html
%
\mode<presentation>
{
  \usetheme{default}      % or try Darmstadt, Madrid, Warsaw, ...
  \usecolortheme{default} % or try albatross, beaver, crane, ...
  \usefonttheme{default}  % or try serif, structurebold, ...
  \setbeamertemplate{navigation symbols}{}
  \setbeamertemplate{caption}[numbered]
} 

\usepackage[brazil]{babel}
\usepackage[utf8]{inputenc}
\usepackage[T1]{fontenc}

\usepackage{tikz}

\title[Transformações]{Transformações Lineares}
\author{MAP 2110 - Diurno}
\institute{IME USP}
\date{19 de maio}

\begin{document}

\begin{frame}
  \titlepage
\end{frame}

\begin{frame}{Espaço $V_n$ como espaço de matrizes}
\begin{center}
  \begin{tikzpicture}
    \draw[red,thick,->] (0,0,0) -- (0,2,0);
    \draw[green,thick,->] (0,0,0) -- (2,0,0);
    \draw[blue,thick,->] (0,0,0) -- (0,0,2);
   \draw[thick,->] (0,0,0) -- (4,2.5,5) node[anchor=west] {
     $\begin{bmatrix}
     x_1 \\ x_2 \\ x_3
   \end{bmatrix}\sim (x_1,x_2,x_3)$};
   \draw[red, dashed] (4,2.5,5) -- (4,0,5);
   \draw[green, dashed] (4,0,5) -- (0,0,5);
   \draw[green, dashed] (0,0,0) -- (4,0,0);
   \draw[blue,dashed] (4,0,5) -- (4,0,0);
   \draw[blue, dashed] (0,0,0) -- (0,0,5);
   % imagem de T_A
   \draw[red,thick,->] (6,0,0) -- (6,2,0);
    \draw[green,thick,->] (6,0,0) -- (8,0,0);
    \draw[blue,thick,->] (6,0,0) -- (6,0,2);
   \draw[thick,->] (6,0,0) -- (7,3,2) node[anchor=west] {
     $A\begin{bmatrix}
     x_1 \\ x_2 \\ x_3
   \end{bmatrix}\sim T_A \mathbf{x}$};
   %\draw (6,-3) grid (12,3);
\end{tikzpicture}
\end{center}
 Se $A$ é uma matriz $m\times n$, ela induz uma transformação
 $T_A:\mathbb{R}^n \to \mathbb{R}^m $ dada por 
 $T_A(\mathbf{x})=A\mathbf{x}$ que é a transformação gerada por $A$.
\end{frame}

\begin{frame}{Exemplo}
  
$A=\begin{pmatrix}
  0 & -2 \\ 2 & 0
\end{pmatrix}$ Então temos:
$$ T_A(x_1, x_2) = \begin{pmatrix}
  0 & -2 \\ 2 & 0
\end{pmatrix} \textcolor{blue}{\begin{bmatrix}
  x_1 \\ x_2
\end{bmatrix}} = \textcolor{red}{\begin{bmatrix}
  -2x_2 \\ 2x_1
\end{bmatrix}} = (-2x_2, 2x_1)$$
\begin{center}
\begin{tikzpicture}
  \draw[gray!60,-latex] (0,0) -- (0,1);
  \draw[gray!60,-latex] (0,0) -- (1,0);
  \draw[blue,-latex] (0,0) -- (1,1);
  \draw[gray!60,-latex] (5,0) -- (5,1);
  \draw[gray!60,-latex] (5,0) -- (6,0);
  \draw[red,-latex] (5,0) -- (3,2);
\end{tikzpicture}
\end{center}
\end{frame}

\begin{frame}{Propriedades das Transformações geradas por matrizes}
  Seja $A$ uma matriz $m\times n$ e identificamos
  $\mathbb{R}^n$ com o conjunto das matrizes de $n$ linhas e
  $1$ coluna de números reais, e da mesma forma $\mathbb{R}^m$ será
  o conjunto das matrizes reais com $m$ linhas e $1$ coluna.
  A tranformação $T_A:\mathbb{R}^n \to \mathbb{R}^m$, como definida acima
  tem as seguintes Propriedades
  \begin{enumerate}
    \item $T_A(\mathbf{x}+\mathbf{y}) = T_A(\mathbf{x}) + T_A(\mathbf{y})$
    \item $T_A(\alpha \mathbf{x})=\alpha T_A(\mathbf{x})$
  \end{enumerate}
  Isso quer dizer que $T_A$ é uma transformação linear.
\end{frame}

\begin{frame}{Exemplo}
  $A=\begin{pmatrix}
    0 & 1 \\ 3 & 1
    \end{pmatrix}$ $\mathbf{x}=\begin{bmatrix}
      1 \\ -1
    \end{bmatrix} \text{ e } \mathbf{y} = \begin{bmatrix}
      2 \\ 1
    \end{bmatrix} \implies \mathbf{x}+ \mathbf{y}=\begin{bmatrix}
      3 \\ 0
    \end{bmatrix}$ Então

  $T_A(\mathbf{x}+\mathbf{y}) = \begin{pmatrix}
    0 & 1 \\ 3 & 1
    \end{pmatrix}\begin{bmatrix}
      3 \\ 0
    \end{bmatrix} = \begin{bmatrix}
      0 \\ 9
    \end{bmatrix}$
$T_A(\mathbf{x}) + T_A(\mathbf{y}) = \begin{pmatrix}
  0 & 1 \\ 3 & 1
  \end{pmatrix}\begin{bmatrix}
    1 \\ -1
  \end{bmatrix} + \begin{pmatrix}
    0 & 1 \\ 3 & 1
    \end{pmatrix}\begin{bmatrix}
      2 \\ 1
    \end{bmatrix}= \begin{bmatrix}
      -1 \\ 2 
    \end{bmatrix} + \begin{bmatrix}
      1 \\ 7
    \end{bmatrix} = \begin{bmatrix}
      0 \\ 9
    \end{bmatrix}$

\end{frame}

\begin{frame}{Obtendo a matriz $A$ a partir da transformação}

  Em $\mathbb{R}^n$ vamos considerar o seguinte conjunto de vetores
  $\{\mathbf{e}_1, \dots, \mathbf{e}_n \}$ dados por

  $$ \mathbf{e}_1 = \begin{bmatrix}
    1 \\ 0 \\\vdots\\ 0 \\ 0
  \end{bmatrix}, \dots, \mathbf{e}_j = \begin{bmatrix}
    0 \\ \vdots \\ 1  \\ \vdots \\ 0
  \end{bmatrix} \text{ com o } 1 \text{ na linha } j$$

  \begin{gather*}
  T_A(\mathbf{e}_j) = A.\mathbf{e}_j = \begin{bmatrix}
    a_{1j} \\ \vdots \\ a_{mj} 
  \end{bmatrix} 
   \text{ com } A=[a_{ij}]  \text{ uma matriz } m\times n
 \end{gather*}

\end{frame}

\begin{frame}{Transformações Lineares} 
  De forma geral $T: \mathbb{R}^n \to \mathbb{R}^m$ é linear
  quando satisfaz:
  \begin{enumerate}
    \item $T(\mathbf{x}+\mathbf{y}) = T(\mathbf{x}) + T(\mathbf{y}) (\forall \mathbf{x}\in \mathbb{R}^n)(\forall \mathbf{y}\in \mathbb{R}^n) $
    \item $T(\alpha \mathbf{x})=\alpha T(\mathbf{x})(\forall \mathbf{x}\in \mathbb{R}^n) (\forall \mathbf{\alpha}\in \mathbb{R})$
  \end{enumerate}
  Então podemos achar a matriz $A m \times n$ que gera $T$ sabendo que $T(\mathbf{e}_j)$ será a $j$-ésima coluna de $A$.
  $$
  \begin{array}{c|ccccc|}
     & T(\mathbf{e}_1)&\cdots &T(\mathbf{e}_j)&\cdots& T(\mathbf{e}_n) \\ \hline
     & a_{11} & \cdots & a_{1j} & \cdots & a_{1n} \\
     & \vdots & \cdots & \vdots & \cdots & \vdots \\
   A=  & a_{i1} & \cdots & a_{ij} & \cdots & a_{in} \\
     & \vdots & \cdots & \vdots & \cdots & \vdots\\
     & a_{m1} & \cdots & a_{mj} & \cdots & a_{mn} \\ \hline
    \end{array}
  $$ 
\end{frame}

\begin{frame}{Exemplo}
  Ache a matriz da transformação que dê o ponto simétrico em relação à reta $r: (0,0)+t(1,1)$
\begin{center}
  \begin{tikzpicture}
    \draw[gray](-1.5,-1.5) grid (4.5,4.5);
    \draw[thick, dashed] (-1.5,-1.5) -- (4.5,4.5);
    \draw[-latex] (0,0) -- (2,0);
    \draw[-latex] (0,0) -- (0,2);
    \draw[-latex, blue] (0,0) -- (1,3);
    \draw[-latex, red] (0,0) -- (3,1);
    \draw[thin,dashed,blue, -latex] (1,3) -- (3,1);
\end{tikzpicture}
\end{center} 
\end{frame}

\begin{frame}
  Note que $T(\mathbf{e}_1)=\mathbf{e}_2$ e $T(\mathbf{e}_2) = \mathbf{e}_1$
  Então
  $$ A= \begin{pmatrix}
    0 & 1 \\ 1 & 0
  \end{pmatrix}$$
\end{frame}


\begin{frame}{Rotação no Plano}
  Qual a matriz de uma Rotação em torno da origem em $\mathbb{R}^2$
  \begin{tikzpicture}
    \draw[gray](-1.5,-1.5) grid (4.5,4.5);
    \draw[-latex] (0,0) -- (2,0);
    \draw[-latex] (0,0) -- (0,2);
    \draw[-latex,blue] (0,0) -- (30:4) node[anchor=west] {$\mathbf{x}$};
    \draw[-latex,red] (0,0) -- (60:4) node[anchor=south] {$A\mathbf{x}$};
    \draw[-latex, blue, dashed] (30:4) arc (30:60:4) node[xshift=0.5cm,yshift=-1.0cm] {$\theta$};
  \end{tikzpicture}

  Note que $T(\mathbf{e}_1)=\cos{\theta}\mathbf{e}_1 + \sin{\theta}\mathbf{e}_2$ e
  $T(\mathbf{e}_2)=-\sin{\theta}\mathbf{e}_1 + \cos{\theta}\mathbf{e}_2$
$$ A = \begin{pmatrix}
  \cos{\theta} & -\sin{\theta} \\
  \sin{\theta} & \cos{\theta}
\end{pmatrix}
$$
\end{frame}


\begin{frame}
  {Projeção ortogonal no plano}
Dada  uma reta que passa pela origem $r: (0,0) + s\mathbf{v}$, achar a 
matriz da projeção ortogonal sobre esta reta

\begin{center}
\begin{tikzpicture}
  \draw[gray](-1.5,-1.5) grid (4.5,4.5);
  \draw[-latex] (0,0) -- (2,0);
  \draw[-latex] (0,0) -- (0,2);
  \draw[dashed] (-1.5,-0.75)--(4,2) node[near end, anchor=south] {$r$};
  \draw[gray] (1,2)--(2,0);
\draw[-latex,blue] (0,0)--(1,2) node[anchor=south] {$\mathbf{x}$};
\draw[-latex,red] (0,0) -- (1.6,0.8) node[anchor=west] {$A\mathbf{x}$};
\end{tikzpicture}
\end{center}
\end{frame}

\begin{frame}
  Agora precisamos lembrar como é a fórmula da projeção, que a gente já fez.
  $$ \text{proj}_{r}(\mathbf{x}) = \frac{\mathbf{x}\cdot\mathbf{v}}{\|\mathbf{v}\|^2}\mathbf{v}$$

  Se colocamos $\mathbf{v}=(v_1,v_2)$ então $\mathbf{e}_1 \cdot \mathbf{v} = v_1$ e  $\mathbf{e}_2 \cdot \mathbf{v} = v_2.$ 
 
  Dessa forma
    $$
    \text{proj}_{r}(\mathbf{e}_1) =
    \begin{bmatrix}
  v_1^2/(v_1^2 + v_2^2) \\ v_1v_2 / (v_1^2 + v_2^2)
\end{bmatrix} \text{ e }\text{proj}_{r}(\mathbf{e}_2) =
\begin{bmatrix}
v_1 v_2/(v_1^2 + v_2^2) \\ v_2^2 / (v_1^2 + v_2^2)
\end{bmatrix} 
$$

$$ A = \frac{1}{(v_1^2+v_2^2)}\begin{pmatrix}
  v_1^2 & v_1v_2 \\ v_1v_2 & v_2^2
\end{pmatrix}
$$
\end{frame}

\begin{frame}{Exemplos em dimensão 3}
  Qual a matriz que define a rotação de uma ângulo $\phi$ em torno do 
  eixo $y$, digamos.
\begin{center}
  \begin{tikzpicture}
    \draw[-latex, thick] (0,0,0) -- (2,0,0) node[anchor=south] {$y$};
    \draw[-latex, thick] (0,0,0) -- (0,2,0) node[anchor=east] {$z$};
    \draw[-latex, thick] (0,0,0) -- (0,0,2) node[anchor=west] {$x$};
    \draw[gray!60] (2,0,0) -- (5,0,0);
    \draw[fill=red] (3,0,0) circle [radius=2pt] node[anchor=east] {$\phi$};
    \draw[green, dashed] (3,0,0) circle [x radius = 0.5cm, y radius= 2cm];

    \draw[gray!60] (3,0) -- ++(240:1.0cm);
    \draw[gray!60] (3,0) -- ++(100:1.7cm);

    \draw[blue,-latex] (0,0,0)--(3.3,0,2.1);
    \draw[red,-latex] (0,0,0)--(3.5,2.4,2.0);
  \end{tikzpicture}
\end{center}
$$ A= \begin{pmatrix}
  \cos(\phi) & 0 & -\sin(\phi)\\
  0 & 1 & 0 \\
  \sin(\phi) & 0 &\cos(\phi)
\end{pmatrix}$$
\end{frame}

\begin{frame}
  Projeção ortogonal sobre um plano gerado pelos vetores $\mathbf{u}$ e $\mathbf{v}$

  \begin{tikzpicture}
    %eixos
    \draw[-latex, thick, gray] (0,0,0) -- (2,0,0) node[anchor=south] {$y$};
    \draw[-latex, thick, gray] (0,0,0) -- (0,2,0) node[anchor=east] {$z$};
    \draw[-latex, thick, gray] (0,0,0) -- (0,0,2) node[anchor=west] {$x$};
    %plano
    
    \shadedraw[red!40, opacity=0.5 ] (0,0,0) -- (3,3,0)-- (7,3,3) -- (4.5,0,3) -- cycle;
    \draw[-latex,green!60!black] (0,0,0) -- (2,2,0);
    \draw[-latex,green!60!black] (0,0,0) -- (3,0,2);

    \draw[-latex,blue] (0,0,0) -- (2,3,5);
    \draw[-latex,red] (0,0,0) -- (4,1,2);
    \draw[red,dashed] (2,3,5) -- (4,1,2);
    \draw (0,0,0) -- (-2,2,3);
  \end{tikzpicture}
\end{frame}

\begin{frame}{Fazendo as contas}
  Vamos chamar 
  $$
  \mathbf{u} = \begin{bmatrix}
    u_1 \\ u_2 \\ u_3
  \end{bmatrix} \text{ e }\mathbf{v} = \begin{bmatrix}
    v_1 \\ v_2 \\ v_3
  \end{bmatrix}$$
  
queremos encontrar  $\alpha$ e $\beta$ tais que
 $$
 \text{proj}(\mathbf{x})=\alpha\mathbf{u} + \beta\mathbf{v} =
 \begin{bmatrix}
  u_1 & v_1 \\ u_2 & v_2 \\ u_3 & v_2
\end{bmatrix} \begin{bmatrix}
  \alpha \\ \beta
\end{bmatrix}
$$
\end{frame}

\begin{frame}{Condição de ortogonalidade}

  O vetor $\mathbf{x} - \text{proj}(\mathbf{x})$ deve ser ortogonal ao 
  plano, e assim ortogonal aos vetores $\mathbf{u}$ e $\mathbf{v}$ que nos 
  fornece as equações
\begin{gather*}
  \mathbf{u}\cdot (\mathbf{x}-\alpha\mathbf{u} -\beta\mathbf{v})=0 \\
  \mathbf{v}\cdot (\mathbf{x}-\alpha\mathbf{u} -\beta\mathbf{v}) =0
\end{gather*}
ou
\begin{gather*}
  \mathbf{u}\cdot \mathbf{x}=\alpha\mathbf{u}\cdot\mathbf{u} -\beta\mathbf{u}\cdot\mathbf{v}=0 \\
  \mathbf{v}\cdot \mathbf{x}=\alpha\mathbf{v}\cdot\mathbf{u} -\beta\mathbf{v}\cdot\mathbf{v} =0
\end{gather*}
  Na forma matricial:
  \begin{gather*} \begin{bmatrix}
    u_1 & u_2 & u_3 \\
    v_1 & v_2 & v_3 
  \end{bmatrix} \mathbf{x} = \begin{bmatrix}
    \mathbf{u}\cdot\mathbf{u} & \mathbf{u}\cdot\mathbf{v} \\
    \mathbf{v}\cdot\mathbf{u} & \mathbf{v}\cdot\mathbf{v}
  \end{bmatrix}\begin{bmatrix}
    \alpha \\ \beta
  \end{bmatrix} \end{gather*}
\end{frame}
\begin{frame}
  Resolvendo a equação para $\alpha$ e $\beta$ temos
  \begin{gather*}
\begin{bmatrix}
  \alpha \\ \beta
\end{bmatrix} = \begin{bmatrix}
  \mathbf{u}\cdot\mathbf{u} & \mathbf{u}\cdot\mathbf{v} \\
  \mathbf{v}\cdot\mathbf{u} & \mathbf{v}\cdot\mathbf{v}
\end{bmatrix}^{-1} \begin{bmatrix}
  u_1 & u_2 & u_3 \\
  v_1 & v_2 & v_3 
\end{bmatrix} \mathbf{x}  \\
\text{proj}(\mathbf{x})=
 \begin{bmatrix}
  u_1 & v_1 \\ u_2 & v_2 \\ u_3 & v_2
\end{bmatrix} \begin{bmatrix}
  \alpha \\ \beta
\end{bmatrix} = \begin{bmatrix}
  u_1 & v_1 \\ u_2 & v_2 \\ u_3 & v_2
\end{bmatrix}\begin{bmatrix}
  \mathbf{u}\cdot\mathbf{u} & \mathbf{u}\cdot\mathbf{v} \\
  \mathbf{v}\cdot\mathbf{u} & \mathbf{v}\cdot\mathbf{v}
\end{bmatrix}^{-1} \begin{bmatrix}
  u_1 & u_2 & u_3 \\
  v_1 & v_2 & v_3 
\end{bmatrix} \mathbf{x} \\
A= \begin{bmatrix}
  u_1 & v_1 \\ u_2 & v_2 \\ u_3 & v_2
\end{bmatrix}\begin{bmatrix}
  \mathbf{u}\cdot\mathbf{u} & \mathbf{u}\cdot\mathbf{v} \\
  \mathbf{v}\cdot\mathbf{u} & \mathbf{v}\cdot\mathbf{v}
\end{bmatrix}^{-1} \begin{bmatrix}
  u_1 & u_2 & u_3 \\
  v_1 & v_2 & v_3 
\end{bmatrix}
\end{gather*}
\end{frame}

\begin{frame}{Verdadeiro ou Falso}
  \begin{block}{1}
Se $T: \mathbb{R}^n \to \mathbb{R}^m$ é uma aplicação linear, então $T(0)=0$
\end{block}
\end{frame}

\begin{frame}{Verdadeiro ou Falso}
  \begin{block}{2}
  Se $F:\mathbb{R}^3 \to \mathbb{R}$ é linear então só pode ser constante igual a zero.
\end{block}
\end{frame}

\begin{frame}{Verdadeiro ou Falso}
  \begin{block}{3}
  Se a matriz $A$ tiver blocos quadrados
  $$ A = \begin{bmatrix}
    A_1 & \mathbf{O} \\
    \mathbf{O} & A_2
  \end{bmatrix}$$ e $A$ é invertível, então cada um dos blocos $A_i$ é invertível.
\end{block}
\end{frame}   

\begin{frame}{Verdadeiro ou Falso}
  \begin{block}{4}
  se $\wedge$ denotar o produto vetorial então temos que
  $$\begin{bmatrix}
    x_1 \\ x_2 \\ x_3 
  \end{bmatrix} \wedge \begin{bmatrix}
    y_1 \\ y_2 \\ y_3
  \end{bmatrix} = \begin{bmatrix}
    0 & -x_3 & x_2 \\
    x_3 & 0 & -x_1 \\
    -x_2  & x_1 &0
  \end{bmatrix}\begin{bmatrix}
    y_1 \\ y_2 \\ y_3
  \end{bmatrix}
  $$
\end{block}
\end{frame}

\begin{frame}{Verdadeiro ou Falso}
  \begin{block}{5}
$A^TA\mathbf{x}=0 \iff A\mathbf{x}=0$
\end{block}
\end{frame}


\end{document}
