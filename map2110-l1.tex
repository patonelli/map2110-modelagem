\documentclass[12pt]{article}
\usepackage{amsfonts, amsmath, amssymb}
\usepackage[brazil]{babel}
\usepackage{graphicx}
\usepackage[T1]{fontenc}
\usepackage{lmodern}

\parindent=0pt

 \addtolength{\textheight}{3.5cm}
 \addtolength{\oddsidemargin}{-1cm}
 \addtolength{\evensidemargin}{-1cm}
 \addtolength{\textwidth}{2cm}
 \addtolength{\topmargin}{-2.0cm}
\newcounter{questao}
\newcommand{\quest}{\stepcounter{questao}{\bf \arabic{questao}.\ }}

\begin{document}
\hrule
 {  \sf  Lista de Exerc�cios- Permuta��es  \hfill \fbox{L1-2016}}
\hrule

\vspace{0.5cm}

\thispagestyle{empty}
\fontsize{14}{16}\selectfont

\quest  Dadas as seguintes permuta��es:
\begin{gather*}
  \sigma =\begin{pmatrix}
    1 & 2 & 3 & 4 & 5 \\
    2 & 1 & 4 & 5 & 3
  \end{pmatrix} \text{ e } \tau =\begin{pmatrix}
    1 & 2 & 3 & 4 & 5 \\
    3 & 5 & 4 & 1 & 2
  \end{pmatrix}
\end{gather*}
Calcular $\sigma\circ\tau,$ $\tau\circ\sigma$, $\sigma^{-1},$ $\tau^{-1},$ $\sigma^{109}\circ\tau^{60}.$

\vspace{0.3cm}

\quest  Mostre que para todo $\sigma,$ permuta��o de $\mathbb{S}_n,$ existe um n�mero inteiro $n>0$ tal que $\sigma^n= \text{id}$.
\vspace{0.3cm}

\quest Quantos s�o os ciclos de ordem $p$ em $\mathbb{S}_n?$

\vspace{0.3cm}

\quest Vamos chamar de suporte de uma permuta��o $\sigma$ de $\mathbb{S}_n$ o conjunto complementar dos pontos fixos em $E_n$.
Mostre que se duas permuta��es t�m suporte disjuntos ent�o elas comutam.

\vspace{0.3cm}

\quest Escrever as permuta��es:
\begin{gather*}
  \sigma =\begin{pmatrix}
    1 & 2 & 3 & 4 & 5 & 6\\
    2 & 1 & 4 & 6 & 3 & 5
  \end{pmatrix} \text{ e } \tau =\begin{pmatrix}
    1 & 2 & 3 & 4 & 5 & 6\\
    3 & 5 & 4 & 1 & 2 & 6
  \end{pmatrix}
\end{gather*}
Como composi��o de ciclos disjuntos.

\quest Verificar se as permuta��es acima s�o pares ou �mpares e d� uma decomposi��o em produto de tansposi��o de cada uma delas.

\quest A matriz de representa��o de uma permuta��o $\sigma \in \mathbb{S}_n$ � a matriz quadrada $n\times n$ dada por $M(\sigma)=(s_{ij})$ onde
\begin{gather*}
  s_{ij}=\left\{
    \begin{array}[h]{cc}
      1 & \text{ se } \sigma{j}=i \\
      0 & \text{ nos outros casos.}
    \end{array}\right.
\end{gather*}
Mostre que vale $M(\sigma).M(\tau) = M(\sigma\circ\tau)$.

\quest Mostre que se $\sigma \in \mathbb{S}_n$ e $\pi \in \mathbb{S}_n$ ent�o o sinal de $\sigma,$ $\sigma^{-1}$ e $\pi\circ\sigma\circ\pi^{-1}$ s�o os mesmos..

\end{document}

%%% Local Variables: 
%%% mode: latex
%%% TeX-master: t
%%% End: 
