\documentclass[12pt]{article}
\usepackage{amsfonts, amsmath, amssymb}
\usepackage[brazil]{babel}
\usepackage{graphicx}
\usepackage[T1]{fontenc}
\usepackage{lmodern}
\usepackage{siunitx}

\parindent=0pt

 \addtolength{\textheight}{3.5cm}
 \addtolength{\oddsidemargin}{-1cm}
 \addtolength{\evensidemargin}{-1cm}
 \addtolength{\textwidth}{2cm}
 \addtolength{\topmargin}{-2.0cm}
\newcounter{questao}
\newcommand{\quest}{\stepcounter{questao}{\bf \arabic{questao}.\ }}

\begin{document}
\hrule
 {  \sf  Lista 4 - Àlgebra de Matrizes  \hfill \fbox{L4-2020}}
\hrule

\vspace{0.5cm}

\thispagestyle{empty}
\fontsize{14}{16}\selectfont

\quest Achar a matriz inversa de 

$$ A = \begin{pmatrix}
   1 & 2 & -1 & 0 \\
   0 & -1 & 1 & 0 \\
   2 & 0 & 1 & 1 \\
   -2 & -1 & 0 &1
\end{pmatrix} $$

\vspace{0.3cm}

\quest Achar a forma escalonada $R$ reduzida de 
$$ A = \begin{pmatrix}
   1 & -2 & 2 \\
   2 & -2 & 0 \\
   -1 & 2 & 1 
\end{pmatrix}$$ e achar uma sequência de matrizes elementares $E_1\dots E_8$ (no máximo 8 mas pode ter menos).
Tal que $E_8.E_7.E_6.E_5.E_4.E_3.E_2.E_1.A=R$ 


\vspace{0.3cm}

\quest Uma matriz $L=[l_{ij}]$ de dimensão $n\times n$ será chamada de \textit{matriz triangular inferior}
se $l_{ii}=1$ ( os elementos da diagonal principal é $1$) e $l_{ij}=0$ se $ i<j $
 ($l_{23}=0$ mas $l_{32}$ pode ser qualquer número). Mostre que o produto de duas matrizes triangular inferior
 também é triangular inferior.

\vspace{0.3cm}

\quest Se $T: \mathbb{R}^3 \to \mathbb{R}^2$ é uma transformação linear
tal que:
\begin{gather*}
   T\begin{bmatrix}
      1 \\ 0 \\ -1
   \end{bmatrix} = \begin{bmatrix}
      2 \\ 3
   \end{bmatrix} \text{ e ]} T\begin{bmatrix}
      2\\ 1 \\ 3
   \end{bmatrix} = \begin{bmatrix}
      -1 \\ 0
   \end{bmatrix} \\
\text{ Encontre o valor de } T\begin{bmatrix}
   8 \\ 3 \\ 7
\end{bmatrix}
\end{gather*}

\vspace{0.3cm}

\quest Em $\mathbb{R}^2$ defino a seguinte operação que faço com os 
vetores $\mathbf{x}=\begin{bmatrix}
   x_1,x_2
\end{bmatrix}:$ Primeiro projeto na reta $r:(0,0) + t(1,1)$, e o resultado giro de 
$\ang{30}$. Escreva a matriz desta transformação.

\end{document}

%%% Local Variables: 
%%% mode: latex
%%% TeX-master: t
%%% End: 
