\documentclass{book}
\usepackage{amsmath}
\usepackage{fontspec}
\usepackage{unicode-math}
\usepackage{polyglossia}

\setdefaultlanguage[variant=brazilian]{portuguese}

\title{Notas de Modelagem Matemática}
\author{Pedro A Tonelli}

\begin{document}
    \maketitle
    \chapter{Números Naturais}

    \section{Axiomas de Peano}

     Quero destacar, neste capítulo, algumas propriedades
    do conjunto dos números naturais que o tornam essencial na modelagem 
    matemática. Ele introduz a noção de ordem, enumeração e quantidade nos 
    outros conjuntos de interesse. Em particular, o conjunto dos números
    naturais será nosso primeiro candidato para expressar matemáticamente a 
    noção de tempo. A aritmética, presente de forma natural em 
  $\mathbb{N}$ irá se generalizar para outros conjuntos numéricos.

  No final do século XIX, o italiano Giuseppe Peano deu uma construção lógica do conjunto dos números naturais. Isto é, uma construção a partir de postulados. Estes axiomas são o resumo das propiedades fundamentais de $\mathbb{N}$

  \begin{itemize}
    \item[A] Existe um elemento especial  em $\mathbb{N}$ que chamaremos de $0$. (Em alguns casos este elemento é o 1).
    \item[B] Existe uma função injetora $S: \mathbb{N} \to \mathbb{N}$, que chamaremos \textit{função sucessor} que satisfaz: $S$ está definida para todo $x\in \mathbb{N}$ e o $0$ não está na imagem de $S$.
    \item[C] (\textbf{Princípio da indução}) Se $K$ é um subconjunto de $\mathbb{N}$ que contém $0$ e se $x\in K$ acarreta que $S(x)\in K$, então $K=\mathbb{N}$
    
  \end{itemize}
\section{Operações algébricas em $\mathbb{N}$}

A brincadeira agora é tentar construir todas as propriedades e operações, com as quais estamos acostumados, do conjunto dos naturais usando apenas os axiomas \textbf{A}, \textbf{B} e \textbf{C}.

Comecemos com a definição de soma: 

\textbf{Definição: } Se $a$ e $b$ são números naturais então definimos:
\begin{eqnarray}
  a + 0 = a & \text{para todo } a \\
  a + S(b) = S(a+b) & \text{ para todo } a \text{ e } b
\end{eqnarray}

Esta é uma definição usando recorrência. A primeira coisa que precisamos entender é que esta soma está definida para  todos os números naturais:
Fazemos assim:
Fixo um número natural $n$ e defino $K_n=\{b\in \mathbb{N}: n+b \text{ está definido }\}$, este conjunto contém $0$, direto da definição e se $b \in K_n$ então $S(b) \in K_n$ e assim pela propriedade \textbf{C} $K_n = \mathbb{N}$  
\end{document}