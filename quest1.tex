\documentclass{beamer}
%
% Choose how your presentation looks.
%
% For more themes, color themes and font themes, see:
% http://deic.uab.es/~iblanes/beamer_gallery/index_by_theme.html
%
\mode<presentation>
{
  \usetheme{default}      % or try Darmstadt, Madrid, Warsaw, ...
  \usecolortheme{default} % or try albatross, beaver, crane, ...
  \usefonttheme{default}  % or try serif, structurebold, ...
  \setbeamertemplate{navigation symbols}{}
  \setbeamertemplate{caption}[numbered]
} 

\usepackage[english]{babel}
\usepackage[utf8]{inputenc}
\usepackage[T1]{fontenc}

\title[]{Falso ou Verdadeiro I}
\author{MAP 2110 - Diurno}
\institute{IME USP}
\date{14 de abril}

\begin{document}

\begin{frame}
  \titlepage
\end{frame}

% Uncomment these lines for an automatically generated outline.
%\begin{frame}{Outline}
%  \tableofcontents
%\end{frame}

\begin{frame}{}
 Seja $X$ um  de $V_n$ então $X\cdot X =0$ se, e somente se $X=0$.
 \textcolor{red}{ Verdadeiro. Pois $X\cdot X= \sum_{i=1}^n x_i^2$ }
\end{frame}

\begin{frame}{}
$$X\cdot Y \leq \|X\|\|Y\|$$

\textcolor{red}{Verdadeiro. É o teorema de Cauchy-Schwarz}
\end{frame}

\begin{frame}
  $\{ (1,2), (1,0), (0,1)\}$ é um conjunto linearmente independente.

  \textcolor{red}{
    Falso pois $(1,2) = (1,0) + 2(0,1)$
  }
\end{frame}

\begin{frame}{}

  $(0-,1, 1)$ pertence ao espaço gerado pelos vetores ${(1,2,0), (0,-1, 0)}$

  \textcolor{red}{
    Falso, pois o espaço gerado pelos vetores tem zero na última coordenada.
  }
\end{frame}

\begin{frame}
  Os vetores $\mathbf{e}_1 = 2\mathbf{i}+\mathbf{j}$ $\mathbf{e_2}= -\mathbf{j}$ e
  $\mathbf{e_3}=\mathbf{j}+\mathbf{k}$ formam uma base de $V_3$

  \textcolor{red}{
    Verdadeiro. Pois são três vetores LI.
    \begin{gather*}
      a\mathbf{e}_1  + b\mathbf{e}_2 +c\mathbf{e}_3 = 0 \implies \\
      2a\mathbf{i} + (a-b+c)\mathbf{j} + c \mathbf{k} = 0 \implies \\
      a=b=c=0
    \end{gather*}
  }
\end{frame}

\begin{frame}
  As retas $r$ e $s$
  com $r$ definida pela equação vetorial  $r: (1,0,1) + \alpha (1,2,-1)$ é paralela 
  à reta $s$ definida pela equação paramétrica
  \begin{gather*}
    x= 2 + 2\lambda \\
    y= -1 + 4\lambda \\
    z= 1  - 2\lambda
  \end{gather*}

  \textcolor{red}{Verdadeiro, pois $(1,2,-1)$ é o vetor de direção das duas retas.}
\end{frame}


\begin{frame}{}
  Se $X$ é um vetor de $V_3$ então $X\times X = 0$ se, e somente se $X=0$

  \textcolor{red}{Falso, $X\times X=0 \forall X\in V_3.$}

\end{frame}

\begin{frame}{}
 Se os vetores $A$ e $B$ de $V_3$ são LI então $\{A,B,A\times B\}$ é uma 
 base de $V_3$

 \textcolor{red}{Verdadeiro.
 \begin{gather*}
   aA +bB + c(A\times B)=0 \\
   aA\cdot(A\times B) +bB\cdot(A\times B) + c(A\times B)\cdot(A\times B)=0\\
   c\|A\times b\|^2 = 0 \implies c=0\\
   aA+bB=0 \implies a=b=0
 \end{gather*}}

\end{frame}

\begin{frame} 
 A equação $3x-y+ 2z =4$ é a equação de um plano que passa pela origem de $V_3$
\textcolor{red}{Falso}
 \end{frame}



\begin{frame}
  O vetor $(3,-1,2)$ é ortogonal ao plano de equação $3x-y+ 2z =4$

  \textcolor{red}{Verdadeiro.}

\end{frame}

\end{document}
