\documentclass[12pt]{article}
\usepackage{amsfonts, amsmath, amssymb}
\usepackage[brazil]{babel}
\usepackage{graphicx}
\usepackage[utf8]{inputenc}
\usepackage[T1]{fontenc}
\usepackage{lmodern}

\parindent=0pt

 \addtolength{\textheight}{3.5cm}
 \addtolength{\oddsidemargin}{-1cm}
 \addtolength{\evensidemargin}{-1cm}
 \addtolength{\textwidth}{2cm}
 \addtolength{\topmargin}{-2.0cm}
\newcounter{questao}
\newcommand{\quest}{\stepcounter{questao}{\bf \arabic{questao}.\ }}

\begin{document}
\hrule
 {  \sf Soluções dos problemas dos slides   \hfill \fbox{}}
\hrule

\vspace{0.5cm}

\thispagestyle{empty}
\fontsize{14}{16}\selectfont

\quest  Considerem os vetores 
$$ \vec{v}_1=\mathbf{i} \text{  }  \vec{v}_2=\mathbf{i} + \mathbf{j} \text{ e }\vec{v}_3 = \mathbf{i + j + 3k} $$

\begin{itemize}
    \item Prove que $\{ \vec{v}_1,\vec{v}_2,\vec{v}_3\}$ é LI
    \item Escreva os vetores $\mathbf{i}$ e $\mathbf{j}$ como combinação linear de $\{ \vec{v}_1,\vec{v}_2,\vec{v}_3\}$
    \item Escreva o vetor $2\mathbf{i} -3\mathbf{j} + 5\mathbf{k}$ como combinação linear de $\{ \vec{v}_1,\vec{v}_2,\vec{v}_3\}$
    \item Prove que $\{ \vec{v}_1,\vec{v}_2,\vec{v}_3\}$ é uma base.
\end{itemize}

\textit{Solução} Este foi o único que fizemos na aula. Para a primeira parte vamos mostrar que se 
$$ \vec{0} = a\vec{v}_1 + b\vec{v}_2 + c\vec{v}_3$$ então devemos ter $a=b=c=0$.

Reescrevemos a fórmula acima usando os vetores canônicos:
$mathbf{i},$ $\mathbf{j}$ e $\mathbf{k}$ e as definições de $\vec{v}_i$ temos:
$$\vec{0} = (a+b+c)\mathbf{i} + (b+c)\mathbf{j} + 3c\mathbf{k} $$
daí concluímos que:
\begin{gather*}
3c = 0 \implies c =0 \\
b+c =0 \implies b+ 0 =0 \implies b=0\\
a+b+c=0 \implies a=0
\end{gather*}
que era o que precisávamos mostrar.

Para a segunda parte vemos que $\mathbf{i} =\vec{v}_1$
e $\mathbf{j} = \vec{v}_2 - \vec{v}_1$ e só para completar temos que $\mathbf{k}=\frac{1}{3}(\vec{v}_3 - \vec{v}_2)$

Assim o vetor  $2\mathbf{i} -3\mathbf{j} + 5\mathbf{k}$ é o mesmo que 
$2(\vec{v}_1) -3(\vec{v}_2 - \vec{v}_1) + \frac{5}{3}(\vec{v}_3 - \vec{v}_2)$ que simplificando se escreve
$$5\vec{v}_1 -\frac{14}{3}\vec{v}_2 + \frac{5}{3}\vec{v}_3 $$
A última parte do exercício pede para provar que o conjunto é uma base. Para ser base precisa mostrar que é um conjunto LI, o que já fizemos, e que o conjunto gera o espaço, isto é que todo vetor de $V_3$ se escreve como combinação linear
de $\vec{v}_i$. Mas simplesmente fazemos como no último exercicio.
$$ a\mathbf{i} + b\mathbf{j} + c\mathbf{k} =a(\vec{v}_1) +b(\vec{v}_2 - \vec{v}_1) + \frac{c}{3}(\vec{v}_3 - \vec{v}_2)$$ etc.



\vspace{0.3cm}

\quest \begin{itemize}
    \item Mostre que os vetores $(\sqrt{3}, 1, 0)$, $(1,\sqrt{3}, 1)$ e $(0,1,\sqrt{3})$ são LI.
    
\item Mostre que os vetores $(\sqrt{2}, 1, 0)$, $(1,\sqrt{2}, 1)$ e $(0,1,\sqrt{3})$ são LD.

\item Encontre todos os valores reais possíveis de $t$ para que os vetores $(t, 1, 0)$, $(1,t, 1)$ e $(0,1,t)$ sejam LD.

\end{itemize}

\textit{Solução:} Vamos mostrar que o primeiro conjunto de vetores é LI. Então
\[ \vec{0} = a(\sqrt{3},1,0) + b(1,\sqrt{3},1) + c (0,1,\sqrt{3}) \] 
Então
\begin{gather*}
\sqrt{3}a+b =0 \\
a + \sqrt{3}b + c =0 \\
b + \sqrt{3}c = 0
\end{gather*}

Note que da primeira e terceira equações concluímos que $a=c.$ A segunda equação fica:  $2a + \sqrt{3}b =0$ podemos multiplicar esta equação por $\frac{\sqrt{3}}{2}$ obtendo
$\sqrt{3}a + \frac{3}{2}b = 0$. Agora subtraindo a primeira equação original temos $\frac{1}{2}b=0$. Então $b=0$ acarreta que $a=0$ e portanto também $c=0$. Pronto.

Porque o outro conjunto é LD. Procedemos como anteriormente com a única diferença de que no lugar de $\sqrt{3}$ temos $\sqrt{2}$. as equações anteriores ficam assim:
\begin{gather*}
\sqrt{2}a+b =0 \\
a + \sqrt{2}b + c =0 \\
b + \sqrt{2}c = 0
\end{gather*}
Novamente concluimos rapidamente que $a=c$ da mesma forma que antes. Então a segunda equação fica:
$2a + \sqrt{2}b =0$ se multiplicamos esta equação por $\frac{\sqrt{2}}{2}$ vemos que esta é exatamente a primeira equação. Ou seja para qualquer valor de $b$ que escolhemos obtemos soluções para $a$ e $c$. Por exemplo,  $ a=-1$ $b=\sqrt{2}$ e $c=-1$ é uma solução do sistema e a combinação linear dos vetores com estes coeficientes produz o vetor nulo.



\vspace{0.3cm}

\quest Se três vetores de $V_n$, $\vec{a},$  $\vec{b}$ e $\vec{c}$ são LI. Verifique se cada uma das afirmações abaixo é verdadeira ou falsa. 
    \begin{itemize}
        \item $\vec{a}+\vec{b}$, $\vec{b}+\vec{c}$ e $\vec{c} + \vec{a}$ formam um conjunto LI.
        
        \item $\vec{a}-\vec{b}$, $\vec{b}+\vec{c}$ e $\vec{c} + \vec{a}$ formam um conjunto LI.
    \end{itemize}

\textit{Solução}
Vamos começar verificando que o segundo item é falso pois
o primeiro vetor do conjunto $\vec{a} -\vec{b}$ é combinação linear dos outros dois. Então os vetores serão LD.
$$ \vec{a}-\vec{b} = (\vec{c}+\vec{a}) - (\vec{b}+\vec{c})$$

O primeiro ítem é verdadeiro pois se
$$\vec{0}= \alpha(\vec{a}+\vec{b}) + \beta(\vec{b}+\vec{c})+\gamma(\vec{c}+\vec{a})= (\alpha +\gamma)\vec{a}+(\alpha+\beta)\vec{b}+(\beta +\gamma)\vec{c}$$ por causa da hipótese de independência de $\{ \vec{a}, \vec{b},\vec{c}\}$. Isto nos dá as equações:
\begin{gather*}
    \alpha + \gamma = 0 \\
    \alpha + \beta = 0 \\
    \beta + \gamma = 0
\end{gather*}
o que trivialmente implica que todos devem ser zero, provando que estes vetores são LI.
Mais prá frente veremos alguns resultados que nos permitirão decidir mais rapidamente sobre a dependência linear dos conjuntos.

\end{document}