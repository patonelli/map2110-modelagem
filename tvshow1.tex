\documentclass{beamer}
%
% Choose how your presentation looks.
%
% For more themes, color themes and font themes, see:
% http://deic.uab.es/~iblanes/beamer_gallery/index_by_theme.html
%
\mode<presentation>
{
  \usetheme{default}      % or try Darmstadt, Madrid, Warsaw, ...
  \usecolortheme{default} % or try albatross, beaver, crane, ...
  \usefonttheme{default}  % or try serif, structurebold, ...
  \setbeamertemplate{navigation symbols}{}
  \setbeamertemplate{caption}[numbered]
} 

\usepackage[english]{babel}
\usepackage[utf8]{inputenc}
\usepackage[T1]{fontenc}

\title{Exercício produto vetorial}
\author{MAP 2110 - Diurno}
\institute{IME USP}
\date{10 de abril}

\begin{document}

\begin{frame}
  \titlepage
\end{frame}

% Uncomment these lines for an automatically generated outline.
%\begin{frame}{Outline}
%  \tableofcontents
%\end{frame}

\section {Produto Vetorial}

\begin{frame}{Definição do Produto Vetorial }
 
 \begin{gather*}
   \vec{a}=(a_1,a_2,a_3)\\
   \vec{b}=(b_1,b_2,b_3)\\
   \vec{a}\times \vec{b}=(a_2b_3-a_3b_2, a_3b_1-a_1b_3, a_1b_2-a_2b_1)
 \end{gather*}

\end{frame}

\begin{frame}{Propriedades}
  \begin{itemize}
    \item $(\vec{a}\times \vec{b}) = - (\vec{b}\times \vec{a})$
    \item $\vec{a}\times(\vec{b} + \vec{c})=\vec{a}\times \vec{b} + \vec{a}\times\vec{c}$
    \item $\alpha(\vec{a}\times \vec{b})=\alpha\vec{a}\times \vec{b}$
    \item $\|\vec{a}\times\vec{b}\|^2 =\|\vec{a}\|^2 \|\vec{b}\|^2 -(\vec{a}\cdot\vec{b})^2$  \end{itemize}

\end{frame}


\begin{frame}{Exercício 15 do Apostol}
  Se $A$ e $B$ são vetores ortogonais e de norma $1$ em $V_3$. Seja $P$ um vetor
  que satisfaz a equação $P\times B = A - P$. Mostre cada uma das afirmações abaixo:
  \begin{itemize}
    \item[a] $P$ é ortogonal a $B$ e mede $\sqrt{2}/2$.
    \item[b] $P\text{, }B$ e $P\times B$ formam uma base de $V_3$.
    \item[c]  $(P\times B)\times B = -P$
    \item[d]  $P= \frac{1}{2}(A-A\times B)$
  \end{itemize}
\end{frame}

\begin{frame}{solução}
\begin{gather*}
  P\times B = A-P \\ \pause 
  (P\times B)\cdot B = (A-P)\cdot B = A\cdot B - P\cdot B \\ \pause 
  P\cdot B = 0
\end{gather*}
\end{frame}

\begin{frame}
  Agora 
  $$ P = A-P\times B $$
  \pause 
  Então
  $$ \|P\|^2 =P\cdot P = A\cdot P $$
\pause 
Lembrando das propriedades do produto vetorial 
$$ \|P\times B\|^2 = \|P\|^2 \|B\|^2 - P\cdot B=\| P \|^2 $$ e
$$ \|P\|^2 = \|P\times B\|^2 = \|A-P\|^2 = \|A\|^2 -2A\cdot P +\|P\|^2 $$
Então $ \cdot P=1/2$ e $\|P\|=\sqrt{2}/2$
\end{frame}

\section{Item b}

\begin{frame}{Item b}
  $P \text{ }B$ e $P\times B$ são vetores ortogonais, LI, portanto formam 
  uma base de $V_3$
\end{frame}

\begin{frame}{Item c}
 $$ (P\times B)\times B = aP+ bB +c(P\times B) $$
 \pause 
 $$ (P\times B)\times B = aP $$
 \pause 
 tomando a norma dos dois lados temos
 $$\|(P\times B)\|\| B \|= |a| \|P\|$$
 ou seja
 $$ a = \pm 1$$
 

\end{frame}

\begin{frame}{item c e d}
  Se $a=1$ teriamos 
  $$(P\times B)\times B=P$$ e
  $$P=(P\times B)\times B= (A-P)\times B = A\times B - (P\times B) = (A\times B) -A +P$$
  oque acarretaria
  $$ A\times B = A \implies A=0$$ 
  contradizendo a hipótese.
\pause
então deve-se ter $a=-1$ e
$$-P = (A\times B) -A +P$$
e dai segue o resultados
\end{frame}

\end{document}
