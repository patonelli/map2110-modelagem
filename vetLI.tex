\documentclass{beamer}
%
% Choose how your presentation looks.
%
% For more themes, color themes and font themes, see:
% http://deic.uab.es/~iblanes/beamer_gallery/index_by_theme.html
%
\mode<presentation>
{
  \usetheme{default}      % or try Darmstadt, Madrid, Warsaw, ...
  \usecolortheme{default} % or try albatross, beaver, crane, ...
  \usefonttheme{default}  % or try serif, structurebold, ...
  \setbeamertemplate{navigation symbols}{}
  \setbeamertemplate{caption}[numbered]
} 

\usepackage[english]{babel}
\usepackage[utf8]{inputenc}
\usepackage[T1]{fontenc}

\title[Vetores LI]{Vetores linearmente independentes e base}
\author{MAP 2110 - Diurno}
\institute{IME USP}
\date{24 de março}

\begin{document}

\begin{frame}
  \titlepage
\end{frame}

% Uncomment these lines for an automatically generated outline.
%\begin{frame}{Outline}
%  \tableofcontents
%\end{frame}

\section{Introdução}

\begin{frame}{Introdução}

Uma família de vetores $\{ \vec{v}_1, \dots, \vec{v}_n \}$ é linearmente independente se, e somente se a única forma de se escrever o vetor nulo $\vec{0}$ como combinação linear é a trivial.
Isto também significa que nenhum vetor $\vec{v}_i$ deste conjunto pode se escrever como combinação linear dos vetores restantes.

Se $\vec{v}$ é um vetor do sub-espaço gerado por $\{ \vec{v}_1 \dots \vec{v}_n \}$ de quantas formas diferentes ele pode ser escrito como combinação linear desses vetores?

\end{frame}

\section{Alguns Exercícios}

\subsection{Exercicios do Apostol}

\begin{frame}{Exercicio 1:}
Considerem os vetores 
$$ \vec{v}_1=\mathbf{i} \text{  }  \vec{v}_2=\mathbf{i} + \mathbf{j} \text{ e }\vec{v}_3 = \mathbf{i + j + 3k} $$

\begin{itemize}
    \item Prove que $\{ \vec{v}_1,\vec{v}_2,\vec{v}_3\}$ é LI
    \item Escreva os vetores $\mathbf{i}$ e $\mathbf{j}$ como combinação linear de $\{ \vec{v}_1,\vec{v}_2,\vec{v}_3\}$
    \item Escreva o vetor $2\mathbf{i} -3\mathbf{j} + 5\mathbf{k}$ como combinação linear de $\{ \vec{v}_1,\vec{v}_2,\vec{v}_3\}$
    \item Prove que $\{ \vec{v}_1,\vec{v}_2,\vec{v}_3\}$ é uma base.
\end{itemize}

\end{frame}

\subsection{Exercicio 2}

\begin{frame}{Exercicio 2}

\begin{itemize}
    \item Mostre que os vetores $(\sqrt{3}, 1, 0)$, $(1,\sqrt{3}, 1)$ e $(0,1,\sqrt{3})$ são LI.
    
\item Mostre que os vetores $(\sqrt{2}, 1, 0)$, $(1,\sqrt{2}, 1)$ e $(0,1,\sqrt{3})$ são LD.

\item Encontre todos os valores reais possíveis de $t$ para que os vetores $(t, 1, 0)$, $(1,t, 1)$ e $(0,1,t)$ sejam LD.

\end{itemize}

\end{frame}

\begin{frame}{Exercício 3}
   Se três vetores de $V_n$, $\vec{a},$  $\vec{b}$ e $\vec{c}$ são LI. Verifique se cada uma das afirmações abaixo é verdadeira ou falsa. 
    \begin{itemize}
        \item $\vec{a}+\vec{b}$, $\vec{b}+\vec{c}$ e $\vec{c} + \vec{a}$ formam um conjunto LI.
        
        \item $\vec{a}-\vec{b}$, $\vec{b}+\vec{c}$ e $\vec{c} + \vec{a}$ formam um conjunto LI.
    \end{itemize}
\end{frame}

\end{document}
